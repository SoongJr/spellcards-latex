%%%
%%% Modernized expl3 format - uses property lists instead of \newcommand
%%% Generated from original spell file
%%%

\begin{spellcard}{sor}{Teleport}{5}
  % Set spell properties using modern expl3 interface
  \spellprop{school}{conjuration}
  \spellprop{subschool}{teleportation}
  \spellprop{descriptor}{NULL}
  \spellprop{castingtime}{1 standard action}
  \spellprop{components}{V}
  \spellprop{range}{personal and touch}
  \spellprop{area}{NULL}
  \spellprop{effect}{NULL}
  \spellprop{targets}{you and touched objects or other touched willing creatures}
  \spellprop{duration}{instantaneous}
  \spellprop{savingthrow}{\textbf{none} and Will negates (object)}
  \spellprop{spellresistance}{\textbf{no} and yes (object)}
  \spellprop{attackroll}{\textbf{none}}

  % Render card information
  \spellcardinfo[0.575]{}
  % QR codes for online resources
  \spellcardqr{https://www.d20pfsrd.com/magic/all-spells/t/teleport}
  \spellcardqr{http://prd.5footstep.de/Grundregelwerk/Zauber/Teleportieren}
  %
  % SPELL DESCRIPTION BEGIN
  Instantly transports to a designated destination no more than 100 miles per caster level distant.
  Interplanar travel is not possible.

  Per three caster levels you can be joined by one willing creature (depending on size).
  All creatures must touch each other and you must touch at least one of them.
  Everyone may bring along as many objects as they can carry.
  Objects held or in use (attended) by another person receive saving throws and spell resistance, creatures do not.

  The accuracy of this spell relies on how well you know the location and layout of the destination.
  Areas of strong physical or magical energy may make teleportation more hazardous or even impossible.

  \begin{longtable}[]{@{}lllll@{}}
    \caption{Roll d\% to determine the outcome}                                                   \\
    \toprule
    Familiarity       & On Target & Off Target & Similar Area & mishap \tabularnewline\midrule
    \endhead{}
    Very familiar     & 01--97    & 98--99     & 100          & --\tabularnewline{}
    Studied carefully & 01--94    & 95--97     & 98--99       & 100\tabularnewline{}
    Seen casually     & 01--88    & 89--94     & 95--98       & 99--100\tabularnewline{}
    Viewed once       & 01--76    & 77--88     & 89--96       & 97--100\tabularnewline{}
    False destination & --        & --         & 81--92       & 93--100\tabularnewline\bottomrule
  \end{longtable}
  \clearpage% explicitly clear page here to prevent underfull vbox warning

  \subsection*{Familiarity}
  \emph{Very familiar} is a place where you have been very often and where you feel at home.

  \emph{Studied carefully} is a place you know well, either because you can currently physically see it or you've been there often.

  \emph{Seen casually} is a place that you have seen more than once but with which you are not very familiar.

  \emph{Viewed once} is a place that you have seen once, possibly using magic such as \emph{scrying.}

  \emph{False destination} does not truly exist or has been so completely altered as to no longer be familiar to you.
  Roll 1d20+80 to obtain results on the table rather than rolling d\%,
  since there is no real destination for you to hope to arrive at or even be off target from.

  \subsection*{Results}
  \emph{On Target}: You appear where you want to be.

  \emph{Off Target}: You appear safely a random distance away from the destination.
  Distance off target is d\% of the distance that was to be traveled.
  The direction off target is determined randomly.

  \emph{Similar Area}: You wind up in an area that's visually or thematically similar to the target area.
  Generally, you appear in the closest similar place within range.
  If no such area exists within the spell's range, the spell simply fails instead.

  \emph{Mishap}: You and anyone else teleporting with you have gotten ``scrambled''.
  You each take 1d10 points of damage, and you reroll on the chart to see where you wind up.
  For these rerolls, roll 1d20+80.

  Each time ``Mishap'' comes up, the characters take more damage and must reroll.
  % SPELL DESCRIPTION END
  %
\end{spellcard}
