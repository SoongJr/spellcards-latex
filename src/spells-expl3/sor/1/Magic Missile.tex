%%%
%%% Modernized expl3 format - uses property lists instead of \newcommand
%%% Generated from original spell file
%%%

\begin{spellcard}{sor}{Magic Missile}{1}
  % Set spell properties using property list
  \spellprop{school}{evocation}
  \spellprop{subschool}{NULL}
  \spellprop{descriptor}{force}
  \spellprop{castingtime}{1 standard action}
  \spellprop{components}{V, S}
  \spellprop{range}{medium (100 ft. + 10 ft./level)}
  \spellprop{area}{NULL}
  \spellprop{effect}{NULL}
  \spellprop{targets}{up to five creatures, no two of which can be more than 15 ft.\ apart}
  \spellprop{duration}{instantaneous}
  \spellprop{savingthrow}{\textbf{none}}
  \spellprop{spellresistance}{yes}
  \spellprop{attackroll}{\textbf{none}}

  % Render card information
  \spellcardinfo{}

  % QR codes for online resources
  \spellcardqr{https://www.d20pfsrd.com/magic/all-spells/m/magic-missile}
  \spellcardqr{http://prd.5footstep.de/Grundregelwerk/Zauber/MagischesGeschoss}

  % Spell description
  A missile of magical energy darts forth from your fingertip and strikes its target,
  dealing 1d4+1 points of force damage.

  The missile strikes unerringly, even if the target is in melee combat,
  so long as it has less than total cover or total concealment.

  Specific parts of a creature can't be singled out.
  Objects are not damaged by the spell.

  For every two caster levels beyond 1\textsuperscript{st}, you gain an additional missile---two at 3\textsuperscript{rd},
  three at 5\textsuperscript{th}, four at 7\textsuperscript{th}, and the maximum of five missiles at 9\textsuperscript{th} level or higher.
  If you shoot multiple missiles, you can have them strike a single creature or several creatures.

  A single missile can strike only one creature.
  You must designate targets before you check for spell resistance or roll damage.
\end{spellcard}
