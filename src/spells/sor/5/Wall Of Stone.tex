% file content generated by convert.sh, meant to be fine-tuned manually (especially the description).

% open a new spellcards environment
\begin{spellcard}{sor}{Wall Of Stone}{5}
  % make the data from TSV accessible for to the LaTeX part:
  \newcommand{\name}{Wall Of Stone}
  \newcommand{\school}{conjuration}
  \newcommand{\subschool}{creation}
  \newcommand{\descriptor}{earth}
  \newcommand{\spelllevel}{5}
  \newcommand{\castingtime}{1 standard action}
  \newcommand{\components}{V, S, M/DF (a small block of granite)}
  \newcommand{\costlycomponents}{0}
  \newcommand{\range}{medium (100 ft. + 10 ft./level)}
  \newcommand{\area}{NULL}
  \newcommand{\effect}{stone wall whose area is up to one 5-ft.\ square/level (S)}
  \newcommand{\targets}{NULL}
  \newcommand{\duration}{instantaneous}
  \newcommand{\dismissible}{0}
  \newcommand{\shapeable}{0}
  \newcommand{\savingthrow}{see text}
  \newcommand{\spellresistance}{\textbf{no}}
  \newcommand{\source}{PFRPG Core}
  \newcommand{\verbal}{1}
  \newcommand{\somatic}{1}
  \newcommand{\material}{1}
  \newcommand{\focus}{0}
  \newcommand{\divinefocus}{1}
  \newcommand{\deity}{NULL}
  \newcommand{\SLALevel}{5}
  \newcommand{\domain}{Earth (5), Fortifications (5)}
  \newcommand{\acid}{0}
  \newcommand{\air}{0}
  \newcommand{\chaotic}{0}
  \newcommand{\cold}{0}
  \newcommand{\curse}{0}
  \newcommand{\darkness}{0}
  \newcommand{\death}{0}
  \newcommand{\disease}{0}
  \newcommand{\earth}{1}
  \newcommand{\electricity}{0}
  \newcommand{\emotion}{0}
  \newcommand{\evil}{0}
  \newcommand{\fear}{0}
  \newcommand{\fire}{0}
  \newcommand{\force}{0}
  \newcommand{\good}{0}
  \newcommand{\languagedependent}{0}
  \newcommand{\lawful}{0}
  \newcommand{\light}{0}
  \newcommand{\mindaffecting}{0}
  \newcommand{\pain}{0}
  \newcommand{\poison}{0}
  \newcommand{\shadow}{0}
  \newcommand{\sonic}{0}
  \newcommand{\water}{0}
  \newcommand{\linktext}{Wall of Stone}
  \newcommand{\id}{620}
  \newcommand{\materialcosts}{NULL}
  \newcommand{\bloodline}{Shaitan (11)}
  \newcommand{\patron}{Mountain (10)}
  \newcommand{\mythictext}{The wall is 1 inch thick per 2 caster levels. The wall's hardness increases to 12.}
  \newcommand{\augmented}{Augmented (7th): If you expend two uses of mythic power, the wall is impassable to ethereal travel and spells such as passwall and phase door. The wall is immune to non-mythic disintegrate, shatter, sympathetic vibration, and other non-mythic magical effects that specifically affect stone (including earthquake, soften earth and stone, and transmute rock to mud).}
  \newcommand{\hauntstatistics}{NULL}
  \newcommand{\ruse}{0}
  \newcommand{\draconic}{0}
  \newcommand{\meditative}{0}
  \newcommand{\urlenglish}{https://www.d20pfsrd.com/magic/all-spells/w/wall-of-stone/}
  \newcommand{\urlgerman}{http://prd.5footstep.de/Grundregelwerk/Zauber/Steinwand}
  % print the tabular information at the top of the card:
  \spellcardinfo{}
  % draw a QR Code pointing at online resources for this spell on the front face:
  \spellcardqr{\urlenglish}
  % ATTENTION: URLs for foreign languages cannot be generated and must be provided by you!
  \spellcardqr{\urlgerman}
  % LaTeX-formatted description of the spell, generated from the HTML-formatted description_formatted column:
  This spell creates a wall of rock that merges into adjoining rock surfaces.
  A \emph{wall of stone} is 1 inch thick per four caster levels and extends up to one 5-foot square per level.
  You can double the wall's area by halving its thickness.
  The wall cannot be conjured so that it occupies the same space as a creature or another object.

  Unlike a \emph{wall of iron}, you can create a \emph{wall of stone} in almost any shape you desire.
  The wall does not need a firm foundation, but it must merge with and be solidly supported by existing stone.
  It can be used to bridge a chasm or as a ramp,but to span more than 20 feet,
  it must be arched and buttressed, which reduces the spell's area by half.
  The wall can be crudely shaped to allow crenellations, battlements,
  and so forth by likewise reducing the area.

  Once the wall is erected, it is a normal, non-magical stone wall and can be destroyed
  by a \emph{disintegrate} spell or any other means that would tear down a normal wall.

  Each 5-foot square of the wall has hardness 8 and 15 hit points per inch of thickness.
  A section of wall whose hit points drop to 0 is breached.
  To break through the wall with a single attack requires a Strength check with DC 20 + 2 per inch of thickness.

  It is possible, but difficult, to trap mobile opponents within or under a \emph{wall of stone}
  Creatures can avoid entrapment with successful Reflex saves.

\end{spellcard}
