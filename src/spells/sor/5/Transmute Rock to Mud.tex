% file content generated by convert.sh, meant to be fine-tuned manually (especially the description).

% open a new spellcards environment
\begin{spellcard}{sor}{Transmute Rock to Mud}{5}
  % make the data from TSV accessible for to the LaTeX part:
  \newcommand{\name}{Transmute Rock to Mud}
  \newcommand{\school}{transmutation}
  \newcommand{\subschool}{NULL}
  \newcommand{\descriptor}{earth}
  \newcommand{\spelllevel}{5}
  \newcommand{\castingtime}{1 standard action}
  \newcommand{\components}{V, S, M/DF (clay and water)}
  \newcommand{\costlycomponents}{0}
  \newcommand{\range}{medium (100 ft. + 10 ft./level)}
  \newcommand{\area}{up to two 10-ft.\ cubes/level}
  \newcommand{\effect}{NULL}
  \newcommand{\targets}{NULL}
  \newcommand{\duration}{permanent; see text}
  \newcommand{\dismissible}{0}
  \newcommand{\shapeable}{1}
  \newcommand{\savingthrow}{see text}
  \newcommand{\spellresistance}{\textbf{no}}
  \newcommand{\source}{PFRPG Core}
  \newcommand{\verbal}{1}
  \newcommand{\somatic}{1}
  \newcommand{\material}{1}
  \newcommand{\focus}{0}
  \newcommand{\divinefocus}{1}
  \newcommand{\deity}{NULL}
  \newcommand{\SLALevel}{5}
  \newcommand{\domain}{NULL}
  \newcommand{\acid}{0}
  \newcommand{\air}{0}
  \newcommand{\chaotic}{0}
  \newcommand{\cold}{0}
  \newcommand{\curse}{0}
  \newcommand{\darkness}{0}
  \newcommand{\death}{0}
  \newcommand{\disease}{0}
  \newcommand{\earth}{1}
  \newcommand{\electricity}{0}
  \newcommand{\emotion}{0}
  \newcommand{\evil}{0}
  \newcommand{\fear}{0}
  \newcommand{\fire}{0}
  \newcommand{\force}{0}
  \newcommand{\good}{0}
  \newcommand{\languagedependent}{0}
  \newcommand{\lawful}{0}
  \newcommand{\light}{0}
  \newcommand{\mindaffecting}{0}
  \newcommand{\pain}{0}
  \newcommand{\poison}{0}
  \newcommand{\shadow}{0}
  \newcommand{\sonic}{0}
  \newcommand{\water}{0}
  \newcommand{\linktext}{Transmute Rock to Mud}
  \newcommand{\id}{576}
  \newcommand{\materialcosts}{NULL}
  \newcommand{\bloodline}{Kobold Sorcerer (11), Kobold (11)}
  \newcommand{\patron}{NULL}
  \newcommand{\mythictext}{NULL}
  \newcommand{\augmented}{NULL}
  \newcommand{\hauntstatistics}{NULL}
  \newcommand{\ruse}{0}
  \newcommand{\draconic}{0}
  \newcommand{\meditative}{0}
  \newcommand{\urlenglish}{https://www.d20pfsrd.com/magic/all-spells/t/transmute-rock-to-mud/}
  \newcommand{\urlsecondary}{http://prd.5footstep.de/Grundregelwerk/Zauber/FelszuSchlammverwandeln}
  % print the tabular information at the top of the card:
  \spellcardinfo{}
  % draw a QR Code pointing at online resources for this spell on the front face:
  \spellcardqr{\urlenglish}
  \spellcardqr{\urlsecondary}
  %
  % SPELL DESCRIPTION BEGIN
  Turn natural, uncut, unworked, non-magical rock of any sort
  into an equal volume of mud, up to 10 feet deep.\\

  \medskip
  A creature unable to fly or otherwise free itself sinks in,
  reducing its speed to 5 feet and its AC by -2.
  Brush or similar material thrown atop the mud can
  support creatures able to climb on top of it. Creatures large enough to
  walk on the bottom can wade through the area at a speed of 5 feet.

  If \emph{transmute rock to mud} is cast upon the ceiling of a cavern or
  tunnel, the mud falls to the floor and spreads out in a pool at a depth
  of 5 feet. The falling mud and the ensuing cave-in deal 8d6 points of
  bludgeoning damage to anyone caught directly beneath the targeted area
  (Reflex save halves).

  Large stone buildings are generally immune to the effect of the spell,
  since it cannot affect worked stone and doesn't reach deep enough to
  undermine such buildings' foundations.
  However, small buildings often rest upon foundations shallow enough
  to be damaged or even partially toppled by this spell.

  The mud remains until a successful \emph{dispel magic} or
  \emph{transmute mud to rock} spell restores its substance -- but not
  necessarily its form.
  Evaporation turns the mud to normal dirt over a period of days,
  depending on exposure to the sun, wind, and normal drainage.

  \medskip
  This spell counters and dispels \emph{transmute mud to rock}.
  % SPELL DESCRIPTION END
  %
\end{spellcard}
