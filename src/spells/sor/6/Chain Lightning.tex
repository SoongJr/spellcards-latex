% file content generated by convert.sh, meant to be fine-tuned manually (especially the description).

% open a new spellcards environment
\begin{spellcard}{sor}{Chain Lightning}{6}
  % make the data from TSV accessible for to the LaTeX part:
  \newcommand{\name}{Chain Lightning}
  \newcommand{\school}{evocation}
  \newcommand{\subschool}{NULL}
  \newcommand{\descriptor}{electricity}
  \newcommand{\spelllevel}{6}
  \newcommand{\castingtime}{1 standard action}
  \newcommand{\components}{V, S, F (a bit of fur; a piece of amber, glass, or a crystal rod; plus one silver pin per caster level)}
  \newcommand{\costlycomponents}{0}
  \newcommand{\range}{long (400 ft. + 40 ft./level)}
  \newcommand{\area}{NULL}
  \newcommand{\effect}{NULL}
  \newcommand{\targets}{one primary target, plus one secondary target/level (each of which must be within 30 ft.\ of the primary target)}
  \newcommand{\duration}{instantaneous}
  \newcommand{\dismissible}{0}
  \newcommand{\shapeable}{0}
  \newcommand{\savingthrow}{Reflex half}
  \newcommand{\spellresistance}{yes}
  \newcommand{\source}{PFRPG Core}
  \newcommand{\verbal}{1}
  \newcommand{\somatic}{1}
  \newcommand{\material}{0}
  \newcommand{\focus}{0}
  \newcommand{\divinefocus}{0}
  \newcommand{\deity}{NULL}
  \newcommand{\SLALevel}{6}
  \newcommand{\domain}{Air (6)}
  \newcommand{\acid}{0}
  \newcommand{\air}{0}
  \newcommand{\chaotic}{0}
  \newcommand{\cold}{0}
  \newcommand{\curse}{0}
  \newcommand{\darkness}{0}
  \newcommand{\death}{0}
  \newcommand{\disease}{0}
  \newcommand{\earth}{0}
  \newcommand{\electricity}{1}
  \newcommand{\emotion}{0}
  \newcommand{\evil}{0}
  \newcommand{\fear}{0}
  \newcommand{\fire}{0}
  \newcommand{\force}{0}
  \newcommand{\good}{0}
  \newcommand{\languagedependent}{0}
  \newcommand{\lawful}{0}
  \newcommand{\light}{0}
  \newcommand{\mindaffecting}{0}
  \newcommand{\pain}{0}
  \newcommand{\poison}{0}
  \newcommand{\shadow}{0}
  \newcommand{\sonic}{0}
  \newcommand{\water}{0}
  \newcommand{\linktext}{Chain Lightning}
  \newcommand{\id}{65}
  \newcommand{\materialcosts}{NULL}
  \newcommand{\bloodline}{Djinni (13), Stormborn (13)}
  \newcommand{\patron}{NULL}
  \newcommand{\mythictext}{This spell deals 1d10 points of damage per caster level (maximum 20d10) and the save DC isn't reduced for secondary targets. Secondary targets have to be within 30 feet of any other target, not necessarily the primary target.}
  \newcommand{\augmented}{NULL}
  \newcommand{\hauntstatistics}{NULL}
  \newcommand{\ruse}{0}
  \newcommand{\draconic}{0}
  \newcommand{\meditative}{0}
  \newcommand{\urlenglish}{https://www.d20pfsrd.com/magic/all-spells/c/chain-lightning/}
  \newcommand{\urlgerman}{http://prd.5footstep.de/Grundregelwerk/Zauber/<german-spell-name>}
  % print the tabular information at the top of the card:
  \spellcardinfo{}
  % draw a QR Code pointing at online resources for this spell on the front face:
  \spellcardqr{\urlenglish}
  % ATTENTION: URLs for foreign languages cannot be generated and must be provided by you!
  %            Set \urlgerman above and activate this line if you want to have it: \spellcardqr{\urlgerman}
  % LaTeX-formatted description of the spell, generated from the HTML-formatted description_formatted column:
  This spell creates an electrical discharge that begins as a single
  stroke commencing from your fingertips. Unlike \emph{lightning bolt},
  \emph{chain lightning} strikes one object or creature initially, then
  arcs to other targets.

  The bolt deals 1d6 points of electricity damage per caster level
  (maximum 20d6) to the primary target. After it strikes, lightning can
  arc to a number of secondary targets equal to your caster level (maximum
  20). The secondary bolts each strike one target and deal as much damage
  as the primary bolt.

  Each target can attempt a Reflex saving throw for half damage.

  The Reflex DC to halve the damage of the secondary bolts is 2 lower than
  the DC to halve the damage of the primary bolt. You choose secondary
  targets as you like, but they must all be within 30 feet of the primary
  target, and no target can be struck more than once. You can choose to
  affect fewer secondary targets than the maximum.

\end{spellcard}
