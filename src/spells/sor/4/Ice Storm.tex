% file content generated by convert.sh, meant to be fine-tuned manually (especially the description).

% open a new spellcards environment
\begin{spellcard}{sor}{Ice Storm}{4}
  % make the data from TSV accessible for to the LaTeX part:
  \newcommand{\name}{Ice Storm}
  \newcommand{\school}{evocation}
  \newcommand{\subschool}{NULL}
  \newcommand{\descriptor}{cold}
  \newcommand{\spelllevel}{4}
  \newcommand{\castingtime}{1 standard action}
  \newcommand{\components}{V, S, M/DF (dust and water)}
  \newcommand{\costlycomponents}{0}
  \newcommand{\range}{long (400 ft. + 40 ft./level)}
  \newcommand{\area}{cylinder (20-ft.\ radius, 40 ft.\ high)}
  \newcommand{\effect}{NULL}
  \newcommand{\targets}{NULL}
  \newcommand{\duration}{1 round/level (D)}
  \newcommand{\dismissible}{1}
  \newcommand{\shapeable}{0}
  \newcommand{\savingthrow}{\textbf{none}}
  \newcommand{\spellresistance}{yes}
  \newcommand{\source}{PFRPG Core}
  \newcommand{\verbal}{1}
  \newcommand{\somatic}{1}
  \newcommand{\material}{1}
  \newcommand{\focus}{0}
  \newcommand{\divinefocus}{1}
  \newcommand{\deity}{NULL}
  \newcommand{\SLALevel}{4}
  \newcommand{\domain}{Water (5), Weather (5)}
  \newcommand{\acid}{0}
  \newcommand{\air}{0}
  \newcommand{\chaotic}{0}
  \newcommand{\cold}{1}
  \newcommand{\curse}{0}
  \newcommand{\darkness}{0}
  \newcommand{\death}{0}
  \newcommand{\disease}{0}
  \newcommand{\earth}{0}
  \newcommand{\electricity}{0}
  \newcommand{\emotion}{0}
  \newcommand{\evil}{0}
  \newcommand{\fear}{0}
  \newcommand{\fire}{0}
  \newcommand{\force}{0}
  \newcommand{\good}{0}
  \newcommand{\languagedependent}{0}
  \newcommand{\lawful}{0}
  \newcommand{\light}{0}
  \newcommand{\mindaffecting}{0}
  \newcommand{\pain}{0}
  \newcommand{\poison}{0}
  \newcommand{\shadow}{0}
  \newcommand{\sonic}{0}
  \newcommand{\water}{0}
  \newcommand{\linktext}{Ice Storm}
  \newcommand{\id}{280}
  \newcommand{\materialcosts}{NULL}
  \newcommand{\bloodline}{NULL}
  \newcommand{\patron}{Winter (6), Storms (8)}
  \newcommand{\mythictext}{The bludgeoning damage increases to 4d8 points of damage and the cold damage increases to 3d6 points of damage. The ground in the area is covered in ice and hailstones, acting as though a grease spell were cast on it. Dealing 5 or more points of fire damage to a square melts the ice and hail, negating the grease effect.}
  \newcommand{\augmented}{Augmented (6th): If you expend two uses of mythic power, one creature in the area is paralyzed (as if by hold person) and gains vulnerability to fire as long as it's paralyzed.}
  \newcommand{\hauntstatistics}{NULL}
  \newcommand{\ruse}{0}
  \newcommand{\draconic}{0}
  \newcommand{\meditative}{0}
  \newcommand{\urlenglish}{https://www.d20pfsrd.com/magic/all-spells/i/ice-storm/}
  \newcommand{\urlgerman}{http://prd.5footstep.de/Grundregelwerk/Zauber/Eissturm}
  % print the tabular information at the top of the card:
  \spellcardinfo{}
  % draw a QR Code pointing at online resources for this spell on the front face:
  \spellcardqr{\urlenglish}
  % ATTENTION: URLs for foreign languages cannot be generated and must be provided by you!
  \spellcardqr{\urlgerman}
  % LaTeX-formatted description of the spell, generated from the HTML-formatted description_formatted column:
  Great magical hailstones pound down upon casting this spell, dealing 3d6
  points of bludgeoning damage and 2d6 points of cold damage to every
  creature in the area. This damage only occurs once, when the spell is
  cast. For the remaining duration of the spell, heavy snow and sleet
  rains down in the area. Creatures inside this area take a -4 penalty on
  Perception skill checks and the entire area is treated as difficult
  terrain. At the end of the duration, the snow and hail disappear,
  leaving no aftereffects (other than the damage dealt).

\end{spellcard}
