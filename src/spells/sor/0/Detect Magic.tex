% file content generated by convert.sh, meant to be fine-tuned manually (especially the description).

% open a new spellcards environment
\begin{spellcard}{sor}{Detect Magic}{0}
  % make the data from TSV accessible for to the LaTeX part:
  \newcommand{\name}{Detect Magic}
  \newcommand{\school}{divination}
  \newcommand{\subschool}{NULL}
  \newcommand{\descriptor}{NULL}
  \newcommand{\spelllevel}{0}
  \newcommand{\castingtime}{1 standard action}
  \newcommand{\components}{V, S}
  \newcommand{\costlycomponents}{0}
  \newcommand{\range}{60 ft.}
  \newcommand{\area}{cone-shaped emanation}
  \newcommand{\effect}{NULL}
  \newcommand{\targets}{NULL}
  \newcommand{\duration}{concentration, up to 1 min./level}
  \newcommand{\dismissible}{1}
  \newcommand{\shapeable}{0}
  \newcommand{\savingthrow}{\textbf{none}}
  \newcommand{\spellresistance}{\textbf{no}}
  \newcommand{\source}{PFRPG Core}
  \newcommand{\verbal}{1}
  \newcommand{\somatic}{1}
  \newcommand{\material}{0}
  \newcommand{\focus}{0}
  \newcommand{\divinefocus}{0}
  \newcommand{\deity}{NULL}
  \newcommand{\SLALevel}{0}
  \newcommand{\domain}{NULL}
  \newcommand{\acid}{0}
  \newcommand{\air}{0}
  \newcommand{\chaotic}{0}
  \newcommand{\cold}{0}
  \newcommand{\curse}{0}
  \newcommand{\darkness}{0}
  \newcommand{\death}{0}
  \newcommand{\disease}{0}
  \newcommand{\earth}{0}
  \newcommand{\electricity}{0}
  \newcommand{\emotion}{0}
  \newcommand{\evil}{0}
  \newcommand{\fear}{0}
  \newcommand{\fire}{0}
  \newcommand{\force}{0}
  \newcommand{\good}{0}
  \newcommand{\languagedependent}{0}
  \newcommand{\lawful}{0}
  \newcommand{\light}{0}
  \newcommand{\mindaffecting}{0}
  \newcommand{\pain}{0}
  \newcommand{\poison}{0}
  \newcommand{\shadow}{0}
  \newcommand{\sonic}{0}
  \newcommand{\water}{0}
  \newcommand{\linktext}{Detect Magic}
  \newcommand{\id}{138}
  \newcommand{\materialcosts}{NULL}
  \newcommand{\bloodline}{NULL}
  \newcommand{\patron}{NULL}
  \newcommand{\mythictext}{NULL}
  \newcommand{\augmented}{NULL}
  \newcommand{\hauntstatistics}{NULL}
  \newcommand{\ruse}{0}
  \newcommand{\draconic}{0}
  \newcommand{\meditative}{0}
  \newcommand{\urlenglish}{https://www.d20pfsrd.com/magic/all-spells/d/detect-magic/}
  \newcommand{\urlsecondary}{http://prd.5footstep.de/Grundregelwerk/Zauber/MagieEntdecken}
  % print the tabular information at the top of the card:
  \spellcardinfo{}
  % draw a QR Code pointing at online resources for this spell on the front face:
  \spellcardqr{\urlenglish}
  \spellcardqr{\urlsecondary}
  % LaTeX-formatted description of the spell, generated from the HTML-formatted description_formatted column:
  You detect magical auras.
  The amount of information revealed depends on how long you study a particular area or subject.

  \emph{1\textsuperscript{st} Round}:
  Presence or absence of magical auras.

  \emph{2\textsuperscript{nd} Round}:
  Number of different magical auras and the power of the most potent aura.

  \emph{3\textsuperscript{rd} Round}:
  The strength and location of each aura.
  If the items or creatures bearing the auras are in line of sight,
  you can make Knowledge (arcana) skill checks to determine the school of magic involved in each.
  (Make one check per aura: DC 15 + spell level, or 15 + 1/2 caster level for a nonspell effect.)
  If the aura emanates from a magic item, you can attempt to identify its properties (see Spellcraft).

  Magical areas, multiple types of magic,
  or strong local magical emanations may distort or conceal weaker auras.

  \subsection*{Aura Strength}
  An aura's power depends on a spell's functioning spell level or an item's caster level;
  see the accompanying table.
  If an aura falls into more than one category, detect magic indicates the stronger of the two.

  \begin{longtable}[]{@{}lllll@{}}
    \caption{Aura power based on spell- or caster-level}                                                                                                                                                          \\
    \toprule
    Spell or Object & Faint                           & Moderate                                     & Strong                                        & Overwhelming \tabularnewline\midrule
    \endhead{}
    Spell effect    & \(\leq{}\)3\textsuperscript{rd} & 4\textsuperscript{th}-6\textsuperscript{th}  & 7\textsuperscript{th}-9\textsuperscript{th}   & \(\geq{}\)10\textsuperscript{th}\tabularnewline{}
    Magic item      & \(\leq{}\)5\textsuperscript{th} & 6\textsuperscript{th}-11\textsuperscript{th} & 12\textsuperscript{th}-20\textsuperscript{th} & \(\geq{}\)21\textsuperscript{st}\tabularnewline\bottomrule
  \end{longtable}

  \subsection*{Lingering Aura}
  A magical aura lingers after its original source dissipates (in the case of a spell)
  or is destroyed (in the case of a magic item).
  If \emph{detect magic} is cast and directed at such a location,
  the spell indicates an aura strength of dim (even weaker than a faint aura).
  How long the aura lingers at this dim level depends on its original power:

  \begin{longtable}[]{@{}ll@{}}
    \toprule
    Original Strength & Duration of Lingering Aura\tabularnewline\midrule
    \endhead{}
    Faint             & 1d6 rounds\tabularnewline{}
    Moderate          & 1d6 minutes\tabularnewline{}
    Strong            & 1d6\(\times{}\)10 minutes\tabularnewline{}
    Overwhelming      & 1d6 days\tabularnewline\bottomrule
  \end{longtable}

  Outsiders and elementals are not magical in themselves, but if they are summoned,
  the conjuration spell registers.
  Each round, you can turn to detect magic in a new area.
  The spell can penetrate barriers, but 1 foot of stone, 1 inch of common metal,
  a thin sheet of lead, or 3 feet of wood or dirt blocks it.

  \emph{Detect magic} can be made permanent with a \emph{permanency} spell.

\end{spellcard}
