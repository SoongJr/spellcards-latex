% file content generated by convert.sh, meant to be fine-tuned manually (especially the description).

% open a new spellcards environment
\begin{spellcard}{sor}{Reverse Gravity}{7}
  % make the data from TSV accessible for to the LaTeX part:
  \newcommand{\name}{Reverse Gravity}
  \newcommand{\school}{transmutation}
  \newcommand{\subschool}{NULL}
  \newcommand{\descriptor}{NULL}
  \newcommand{\spelllevel}{7}
  \newcommand{\castingtime}{1 standard action}
  \newcommand{\components}{V, S, M/DF (lodestone and iron filings)}
  \newcommand{\costlycomponents}{0}
  \newcommand{\range}{medium (100 ft. + 10 ft./level)}
  \newcommand{\area}{up to one 10-ft.\ cube/level (S)}
  \newcommand{\effect}{NULL}
  \newcommand{\targets}{NULL}
  \newcommand{\duration}{1 round/level (D)}
  \newcommand{\dismissible}{1}
  \newcommand{\shapeable}{1}
  \newcommand{\savingthrow}{see text}
  \newcommand{\spellresistance}{\textbf{no}}
  \newcommand{\source}{PFRPG Core}
  \newcommand{\verbal}{1}
  \newcommand{\somatic}{1}
  \newcommand{\material}{1}
  \newcommand{\focus}{0}
  \newcommand{\divinefocus}{1}
  \newcommand{\deity}{NULL}
  \newcommand{\SLALevel}{7}
  \newcommand{\domain}{Void (7)}
  \newcommand{\acid}{0}
  \newcommand{\air}{0}
  \newcommand{\chaotic}{0}
  \newcommand{\cold}{0}
  \newcommand{\curse}{0}
  \newcommand{\darkness}{0}
  \newcommand{\death}{0}
  \newcommand{\disease}{0}
  \newcommand{\earth}{0}
  \newcommand{\electricity}{0}
  \newcommand{\emotion}{0}
  \newcommand{\evil}{0}
  \newcommand{\fear}{0}
  \newcommand{\fire}{0}
  \newcommand{\force}{0}
  \newcommand{\good}{0}
  \newcommand{\languagedependent}{0}
  \newcommand{\lawful}{0}
  \newcommand{\light}{0}
  \newcommand{\mindaffecting}{0}
  \newcommand{\pain}{0}
  \newcommand{\poison}{0}
  \newcommand{\shadow}{0}
  \newcommand{\sonic}{0}
  \newcommand{\water}{0}
  \newcommand{\linktext}{Reverse Gravity}
  \newcommand{\id}{452}
  \newcommand{\materialcosts}{NULL}
  \newcommand{\bloodline}{Starsoul (15)}
  \newcommand{\patron}{Trickery (14)}
  \newcommand{\mythictext}{Creatures in the area or that enter the area must succeed at a Fortitude save or be nauseated.}
  \newcommand{\augmented}{Augmented (8\textsuperscript{th}): If you expend three uses of mythic power, once per round as a move action you may select one secured creature (one that succeeded at its Reflex save) or attached object (such as a tree or cottage) and force it to attempt a Fortitude save against the spell. The selected creature or object can weigh no more than 100 pounds per caster level. If it fails the save, it's pulled free and falls upward.}
  \newcommand{\hauntstatistics}{NULL}
  \newcommand{\ruse}{0}
  \newcommand{\draconic}{0}
  \newcommand{\meditative}{0}
  \newcommand{\urlenglish}{https://www.d20pfsrd.com/magic/all-spells/r/reverse-gravity/}
  \newcommand{\urlsecondary}{http://prd.5footstep.de/Grundregelwerk/Zauber/Schwerkraftumkehren}
  % print the tabular information at the top of the card:
  \spellcardinfo{}
  % draw a QR Code pointing at online resources for this spell on the front face:
  \spellcardqr{\urlenglish}
  \spellcardqr{\urlsecondary}
  % LaTeX-formatted description of the spell, generated from the HTML-formatted description_formatted column:
  This spell reverses gravity in an area, causing unattached objects and creatures in the area to fall upward
  and reach the top of the area in 1 round.
  If a solid object (such as a ceiling) is encountered in this fall, falling objects and creatures strike it
  in the same manner as they would during a normal downward fall.
  If an object or creature reaches the top of the area without striking anything,
  it remains there, oscillating slightly, until the spell ends.
  At the end of the spell duration, affected objects and creatures fall downward.

  Provided it has something to hold onto, a creature caught in the area
  can attempt a Reflex save to secure itself when the spell strikes.

  Creatures who can fly or levitate can keep themselves from falling.

\end{spellcard}
