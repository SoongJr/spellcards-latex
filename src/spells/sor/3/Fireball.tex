% file content generated by convert.sh, meant to be fine-tuned manually (especially the description).

% open a new spellcards environment
\begin{spellcard}{sor}{Fireball}{3}
  % make the data from TSV accessible for to the LaTeX part:
  \newcommand{\name}{Fireball}
  \newcommand{\school}{evocation}
  \newcommand{\subschool}{NULL}
  \newcommand{\descriptor}{fire}
  \newcommand{\spelllevel}{3}
  \newcommand{\castingtime}{1 standard action}
  \newcommand{\components}{V, S, M (a ball of bat guano and sulfur)}
  \newcommand{\costlycomponents}{0}
  \newcommand{\range}{long (400 ft. + 40 ft./level)}
  \newcommand{\area}{20-ft.-radius spread}
  \newcommand{\effect}{NULL}
  \newcommand{\targets}{NULL}
  \newcommand{\duration}{instantaneous}
  \newcommand{\dismissible}{0}
  \newcommand{\shapeable}{0}
  \newcommand{\savingthrow}{Reflex half}
  \newcommand{\spellresistance}{yes}
  \newcommand{\source}{PFRPG Core}
  \newcommand{\verbal}{1}
  \newcommand{\somatic}{1}
  \newcommand{\material}{1}
  \newcommand{\focus}{0}
  \newcommand{\divinefocus}{0}
  \newcommand{\deity}{NULL}
  \newcommand{\SLALevel}{3}
  \newcommand{\domain}{Fire (3)}
  \newcommand{\acid}{0}
  \newcommand{\air}{0}
  \newcommand{\chaotic}{0}
  \newcommand{\cold}{0}
  \newcommand{\curse}{0}
  \newcommand{\darkness}{0}
  \newcommand{\death}{0}
  \newcommand{\disease}{0}
  \newcommand{\earth}{0}
  \newcommand{\electricity}{0}
  \newcommand{\emotion}{0}
  \newcommand{\evil}{0}
  \newcommand{\fear}{0}
  \newcommand{\fire}{1}
  \newcommand{\force}{0}
  \newcommand{\good}{0}
  \newcommand{\languagedependent}{0}
  \newcommand{\lawful}{0}
  \newcommand{\light}{0}
  \newcommand{\mindaffecting}{0}
  \newcommand{\pain}{0}
  \newcommand{\poison}{0}
  \newcommand{\shadow}{0}
  \newcommand{\sonic}{0}
  \newcommand{\water}{0}
  \newcommand{\linktext}{Fireball}
  \newcommand{\id}{207}
  \newcommand{\materialcosts}{NULL}
  \newcommand{\bloodline}{Efreeti (7)}
  \newcommand{\patron}{Elements (6)}
  \newcommand{\mythictext}{The damage dealt increases to 1d10 points of fire damage per caster level (maximum 10d10). Any creature that fails its Reflex saving throw catches on fire (Core Rulebook 444), taking 2d6 points of fire damage each round until the fire is extinguished. Attempts to extinguish this fire use the spell's save DC.}
  \newcommand{\augmented}{Augmented (6th): If you expend two uses of mythic power, the maximum damage increases to 20d10, the area increases to a 40-foot radius spread, and any fire damage dealt by the spell bypasses fire resistance and fire immunity.}
  \newcommand{\hauntstatistics}{NULL}
  \newcommand{\ruse}{0}
  \newcommand{\draconic}{0}
  \newcommand{\meditative}{0}
  % print the tabular information at the top of the card:
  \spellcardinfo{}
  % LaTeX-formatted description of the spell, generated from the HTML-formatted description_formatted column:
  A \emph{fireball} spell generates a searing explosion of flame that
  de\-to\-nates with a low roar and deals 1d6 points of fire damage per caster
  level (maximum 10d6) to every creature within the area.

  Unattended objects also take this damage. The explosion creates almost
  no pressure.

  You point your finger and determine the range (distance and height) at
  which the \emph{fireball} is to burst. A glowing, pea-sized bead streaks
  from the pointing digit and, unless it impacts upon a material body or
  solid barrier prior to attaining the prescribed range, blossoms into the
  \emph{fireball} at that point. An early impact results in an early
  detonation. If you attempt to send the bead through a narrow passage,
  such as through an arrow slit, you must ``hit'' the opening with a ranged
  touch attack, or else the bead strikes the barrier and detonates
  prematurely.

  The \emph{fireball} sets fire to combustibles and damages objects in the
  area. It can melt metals with low melting points, such as lead, gold,
  copper, silver, and bronze. If the damage caused to an interposing
  barrier shatters or breaks through it, the \emph{fireball} may continue
  beyond the barrier if the area permits; otherwise it stops at the
  barrier just as any other spell effect does.

\end{spellcard}
