%%%
%%% file content generated by spell_card_generator.py,
%%% meant to be fine-tuned manually (especially the description).
%%%
%
% open a new spellcards environment
\begin{spellcard}{sor}{Dispel Magic}{3}
  % make the data from TSV accessible for to the LaTeX part:
  \newcommand{\name}{Dispel Magic}
  \newcommand{\school}{abjuration}
  \newcommand{\subschool}{NULL}
  \newcommand{\descriptor}{NULL}
  \newcommand{\spelllevel}{3}
  \newcommand{\castingtime}{1 standard action}
  \newcommand{\components}{V, S}
  \newcommand{\costlycomponents}{0}
  \newcommand{\range}{medium (100 ft. + 10 ft./level)}
  \newcommand{\area}{one spellcaster, creature, or object}
  \newcommand{\effect}{NULL}
  \newcommand{\targets}{one spellcaster, creature, or object}
  \newcommand{\duration}{instantaneous}
  \newcommand{\dismissible}{0}
  \newcommand{\shapeable}{0}
  \newcommand{\savingthrow}{\textbf{none}}
  \newcommand{\spellresistance}{\textbf{no}}
  \newcommand{\source}{PFRPG Core}
  \newcommand{\verbal}{1}
  \newcommand{\somatic}{1}
  \newcommand{\material}{0}
  \newcommand{\focus}{0}
  \newcommand{\divinefocus}{0}
  \newcommand{\deity}{NULL}
  \newcommand{\SLALevel}{3}
  \newcommand{\domain}{Magic (3), Entropy (3)}
  \newcommand{\acid}{0}
  \newcommand{\air}{0}
  \newcommand{\chaotic}{0}
  \newcommand{\cold}{0}
  \newcommand{\curse}{0}
  \newcommand{\darkness}{0}
  \newcommand{\death}{0}
  \newcommand{\disease}{0}
  \newcommand{\earth}{0}
  \newcommand{\electricity}{0}
  \newcommand{\emotion}{0}
  \newcommand{\evil}{0}
  \newcommand{\fear}{0}
  \newcommand{\fire}{0}
  \newcommand{\force}{0}
  \newcommand{\good}{0}
  \newcommand{\languagedependent}{0}
  \newcommand{\lawful}{0}
  \newcommand{\light}{0}
  \newcommand{\mindaffecting}{0}
  \newcommand{\pain}{0}
  \newcommand{\poison}{0}
  \newcommand{\shadow}{0}
  \newcommand{\sonic}{0}
  \newcommand{\water}{0}
  \newcommand{\linktext}{Dispel Magic}
  \newcommand{\id}{159}
  \newcommand{\materialcosts}{NULL}
  \newcommand{\bloodline}{Arcane (7)}
  \newcommand{\patron}{NULL}
  \newcommand{\mythictext}{When used as a targeted dispel, this spell can end two spells affecting the target instead of just one. If the targeted dispel successfully dispels at least one spell, you heal 1d4 points of damage for every spell level of the dispelled spell. If you dispel two spells, this healing applies only to the highest-level spell dispelled. If you use this spell to counterspell, roll your dispel check twice and take the higher result.}
  \newcommand{\augmented}{NULL}
  \newcommand{\hauntstatistics}{NULL}
  \newcommand{\ruse}{0}
  \newcommand{\draconic}{0}
  \newcommand{\meditative}{0}
  \newcommand{\urlenglish}{https://www.d20pfsrd.com/magic/all-spells/d/dispel-magic/}
  \newcommand{\urlsecondary}{http://prd.5footstep.de/Grundregelwerk/Zauber/MagieBannen}
  % print the tabular information at the top of the card:
  \spellcardinfo{}
  % draw a QR Code pointing at online resources for this spell on the front face:
  \spellcardqr{\urlenglish}
  \spellcardqr{\urlsecondary}
  %
  % SPELL DESCRIPTION BEGIN
  You can use \emph{dispel magic} to end ongoing spells that were cast on a creature, object or area,
  to temporarily suppress the magical abilities of a magic item, or to counter another spellcaster's spell.
  A dispelled spell ends as if its duration had expired.
  Some spells, as detailed in their descriptions, cannot be defeated by \emph{dispel magic}.

  \emph{Dispel magic} can dispel (but not counter) spell-like effects just as it does spells.
  The result of a spell with an instantaneous duration cannot be dispelled,
  because the magical effect is already over before the \emph{dispel magic} can take effect.

  You choose to use \emph{dispel magic} in one of two ways: a counterspell or targeted dispel.

  \subsection*{Counterspell}
  When \emph{dispel magic} is used in this way, the spell targets a spellcaster and is cast as a counterspell.

  Unlike a true counterspell, however, \emph{dispel magic} may not work;
  you must make a dispel check (1d20 + your caster level)
  to counter the other spellcaster's spell (DC = 11 + the spellcaster's level).

  \clearpage{}\subsection*{Targeted Dispel}
  You make one dispel check (1d20 + your caster level) and compare that to the spell with highest caster level
  (DC = 11 + the spell's caster level) currently affecting a creature or object.

  If successful, that spell ends.

  If not, compare the same result to the spell with the next highest caster level.
  Repeat this process until you have dispelled one spell affecting the target,
  or you have failed to dispel every spell.

  Please check the rulebook for examples.

  You can also use a targeted dispel to end one specific spell affecting the target or one spell affecting an area
  (such as a \emph{wall of fire}).
  You must name the specific spell effect to be targeted in this way.
  No other spells or effects on the target are dispelled if your check is not high enough to end the targeted effect.

  If you target an object or creature that is the effect of an ongoing spell
  (such as a monster summoned by \emph{summon monster}),
  you make a dispel check to end the spell that conjured the object or creature.

  If the object that you target is a magic item,
  you make a dispel check against the item's caster level (DC = 11 + the item's caster level).
  If you succeed, all the item's magical properties are suppressed for 1d4 rounds.
  A suppressed item is considered nonmagical for the duration of the effect.
  An interdimensional opening (such as a \emph{bag of holding}) is temporarily closed.
  Any non-magical properties are unchanged.
  Artifacts and deities are unaffected by mortal magic such as this.

  You automatically succeed on your dispel check against any spell that you cast yourself.
  % SPELL DESCRIPTION END
  %
\end{spellcard}
