% file content generated by convert.sh, meant to be fine-tuned manually (especially the description).

% open a new spellcards environment
\begin{spellcard}{wiz}{Wall Of Force}{5}
  % make the data from TSV accessible for to the LaTeX part:
  \newcommand{\name}{Wall Of Force}
  \newcommand{\school}{evocation}
  \newcommand{\subschool}{NULL}
  \newcommand{\descriptor}{force}
  \newcommand{\spelllevel}{5}
  \newcommand{\castingtime}{1 standard action}
  \newcommand{\components}{V, S, M (powdered quartz)}
  \newcommand{\costlycomponents}{0}
  \newcommand{\range}{close (25 ft. + 5 ft./2 levels)}
  \newcommand{\area}{NULL}
  \newcommand{\effect}{wall whose area is up to one 10-ft.\ square/level}
  \newcommand{\targets}{NULL}
  \newcommand{\duration}{1 round /level (D)}
  \newcommand{\dismissible}{1}
  \newcommand{\shapeable}{0}
  \newcommand{\savingthrow}{none}
  \newcommand{\spellresistance}{no}
  \newcommand{\source}{PFRPG Core}
  \newcommand{\verbal}{1}
  \newcommand{\somatic}{1}
  \newcommand{\material}{1}
  \newcommand{\focus}{0}
  \newcommand{\divinefocus}{0}
  \newcommand{\deity}{NULL}
  \newcommand{\SLALevel}{5}
  \newcommand{\domain}{Isolation (6)}
  \newcommand{\acid}{0}
  \newcommand{\air}{0}
  \newcommand{\chaotic}{0}
  \newcommand{\cold}{0}
  \newcommand{\curse}{0}
  \newcommand{\darkness}{0}
  \newcommand{\death}{0}
  \newcommand{\disease}{0}
  \newcommand{\earth}{0}
  \newcommand{\electricity}{0}
  \newcommand{\emotion}{0}
  \newcommand{\evil}{0}
  \newcommand{\fear}{0}
  \newcommand{\fire}{0}
  \newcommand{\force}{1}
  \newcommand{\good}{0}
  \newcommand{\languagedependent}{0}
  \newcommand{\lawful}{0}
  \newcommand{\light}{0}
  \newcommand{\mindaffecting}{0}
  \newcommand{\pain}{0}
  \newcommand{\poison}{0}
  \newcommand{\shadow}{0}
  \newcommand{\sonic}{0}
  \newcommand{\water}{0}
  \newcommand{\linktext}{Wall of Force}
  \newcommand{\id}{598}
  \newcommand{\materialcosts}{NULL}
  \newcommand{\bloodline}{NULL}
  \newcommand{\patron}{NULL}
  \newcommand{\mythictext}{The wall's hardness increases to 40, and its hit points increase to 30 per caster level. A non-mythic disintegrate spell or rod of cancellation negates a 10-foot-square section of a mythic wall of force for 1 round, after which the wall reforms at full strength. One side of the wall (chosen by you) repels creatures within 5 feet as a repulsion spell (using the DC wall of force would have if it allowed a saving throw).}
  \newcommand{\augmented}{NULL}
  \newcommand{\hauntstatistics}{NULL}
  \newcommand{\ruse}{0}
  \newcommand{\draconic}{0}
  \newcommand{\meditative}{0}
  % print the tabular information at the top of the card:
  \spellcardinfo{}
  % LaTeX-formatted description of the spell, generated from the HTML-formatted description_formatted column:
  A \emph{wall of force} creates an invisible wall of pure force. The wall
  cannot move and is not easily destroyed. A \emph{wall of force} is
  immune to \emph{dispel magic}, although a mage's disjunction can still
  dispel it.

  A \emph{wall of force} can be damaged by spells as normal, except for
  \emph{disintegrate}, which automatically destroys it. It can be damaged
  by weapons and supernatural abilities, but a \emph{wall of force} has
  hardness 30 and a number of hit points equal to 20 per caster level.
  Contact with a \emph{sphere of annihilation} or \emph{rod of
    cancellation} instantly destroys a \emph{wall of force}.

  Breath weapons and spells cannot pass through a \emph{wall of force} in
  either direction, although \emph{dimension door}, \emph{teleport}, and
  si\-mi\-lar effects can bypass the barrier. It blocks ethereal creatures as
  well as material ones (though ethereal creatures can usually circumvent
  the wall by going around it, through material floors and ceilings).\\
  Gaze attacks can operate through a \emph{wall of force}.

  The caster can form the wall into a flat, vertical plane whose area is
  up to one 10-foot square per level. The wall must be continuous and
  unbroken when formed. If its surface is broken by any object or
  creature, the spell fails.

  \emph{Wall of force} can be made permanent with a \emph{permanency}
  spell.

\end{spellcard}
