
\documentclass[%
  a4paper,
  twoside,      % one side is the front, the other the back of cards. With \cleardoublepage we can ensure a spell starts on the front of a new card.
  DIV=14,       % reduce margins to account for the reduced space on the smaller cards
  fontsize=12pt,% increase font size to make the text more readable when scaled down to four cards per A4 page
  BCOR=0mm,     % no binding correction necessary
  % draft=true,   % helps to point out overfull boxes etc.
]{scrartcl}

\usepackage[utf8]{inputenc}
\usepackage{calc}               % for floating point calculations
\usepackage{ifthen}             % \ifthenelse command
\usepackage{booktabs}           % nicer table formatting options
\usepackage{tabularx}           % for tables that can span the whole page width
% packages used indirectly by pandoc converting from HTML code:
\usepackage{longtable}

% include file to make the cards print four on a page, double-sided, to be easily cut and end up sorted:
% (to address over-/underfull boxes, it can be useful to disable this so it's easier to see which spell you are working on right now)
% % this takes each page of the document, sizes it down to DIN A6 and places it on a DIN A4 page
% in a way that they can be cut and folded over in a way that it all ends up sorted correctly.
% lots of rotation and reordering magic going on.
%\url{https://tex.stackexchange.com/q/638802/86}
%Adapted from \url{http://tex.stackexchange.com/q/279042/86}
\usepackage{pgfmorepages}
\usepackage{atbegshi}           % for using \AtBeginShipout, required to draw lines over all pages

\pgfpagesdeclarelayout{8 on 2, book format}
{%
  \edef\pgfpageoptionheight{\the\paperheight}
  \edef\pgfpageoptionwidth{\the\paperwidth}
  \def\pgfpageoptionborder{0pt}
  \def\pgfpageoptionfirstshipout{1}
}%
{%
  \pgfpagesphysicalpageoptions%
  {%
    logical pages=8,%
    physical pages=2,%
    physical height=\pgfpageoptionheight,%
    physical width=\pgfpageoptionwidth,%
    current logical shipout=\pgfpageoptionfirstshipout%
  }
  \pgfpagesphysicalpage{1}{}
  \pgfpageslogicalpageoptions{1}
  {%
    rotation=0,
    border shrink=\pgfpageoptionborder,%
    resized width=.5\pgfphysicalwidth,%
    resized height=.5\pgfphysicalheight,%
    center=\pgfpoint{.25\pgfphysicalwidth}{.75\pgfphysicalheight}%
  }%
  \pgfpageslogicalpageoptions{8}
  {%
    rotation=0,
    border shrink=\pgfpageoptionborder,%
    resized width=.5\pgfphysicalwidth,%
    resized height=.5\pgfphysicalheight,%
    center=\pgfpoint{.75\pgfphysicalwidth}{.75\pgfphysicalheight}%
  }%
  \pgfpageslogicalpageoptions{4}
  {%
    rotation=180,
    border shrink=\pgfpageoptionborder,%
    resized width=.5\pgfphysicalwidth,%
    resized height=.5\pgfphysicalheight,%
    center=\pgfpoint{.25\pgfphysicalwidth}{.25\pgfphysicalheight},%
  }%
  \pgfpageslogicalpageoptions{5}
  {%
    rotation=180,
    border shrink=\pgfpageoptionborder,%
    resized width=.5\pgfphysicalwidth,%
    resized height=.5\pgfphysicalheight,%
    center=\pgfpoint{.75\pgfphysicalwidth}{.25\pgfphysicalheight},%
  }%
  \pgfpagesphysicalpage{2}{}
  \pgfpageslogicalpageoptions{7}
  {%
    rotation=0,
    border shrink=\pgfpageoptionborder,%
    resized width=.5\pgfphysicalwidth,%
    resized height=.5\pgfphysicalheight,%
    center=\pgfpoint{.25\pgfphysicalwidth}{.75\pgfphysicalheight}%
  }%
  \pgfpageslogicalpageoptions{2}
  {%
    rotation=0,
    border shrink=\pgfpageoptionborder,%
    resized width=.5\pgfphysicalwidth,%
    resized height=.5\pgfphysicalheight,%
    center=\pgfpoint{.75\pgfphysicalwidth}{.75\pgfphysicalheight}%
  }%
  \pgfpageslogicalpageoptions{6}
  {%
    rotation=180,
    border shrink=\pgfpageoptionborder,%
    resized width=.5\pgfphysicalwidth,%
    resized height=.5\pgfphysicalheight,%
    center=\pgfpoint{.25\pgfphysicalwidth}{.25\pgfphysicalheight},%
  }%
  \pgfpageslogicalpageoptions{3}
  {%
    rotation=180,
    border shrink=\pgfpageoptionborder,%
    resized width=.5\pgfphysicalwidth,%
    resized height=.5\pgfphysicalheight,%
    center=\pgfpoint{.75\pgfphysicalwidth}{.25\pgfphysicalheight},%
  }%
}

\AtBeginDocument{
  \pgfpagesuselayout{8 on 2, book format} % use the layout we set up above
}

\AtBeginShipout{
  \AtBeginShipoutUpperLeft{
    % draw cut lines to show where to cut the paper
    \begin{tikzpicture}[remember picture, overlay]
      \draw[lightgray, dashed] (current page.north east) -- (current page.south east) -- (current page.south west);
    \end{tikzpicture}
  }
}

% when being cardified, ensure a new card's front face does not start on the backside of the previous card:
\renewcommand{\clearcard}{\cleardoublepage}


% include templates for composing spell cards
% this file contains templates to show spell cards.
% They are be used by convert.sh to create spell cards from the Spell-DB,
% but the user is encouraged to modify those files to best suit their needs
% (fit spells on the cards nicely, rephrase ambigious descriptions,
% include information from other spells for those "works like X" spells, etc.)


% %%%%%%%%%%%%%%%%%%%%%%%%%%%%%%%%%%%%%%%%%%%%%%%%%%%%%%%%%%%%%%%
% environment for formatting a spell card (title-bar, header/footer, etc.)
% %%%%%%%%%%%%%%%%%%%%%%%%%%%%%%%%%%%%%%%%%%%%%%%%%%%%%%%%%%%%%%%
% function for drawing the spell marker
% draw a marker on the right side of the page to quickly flip through the cards
% and find the spell level you are looking for.
% When cardified, this edge will be at the center of a DIN A4 page,
% so there is no problem printing "all the way to the edge".
% The intention is for the user to color the marker with a highlighter after cutting the pages free.
\newcommand{\drawspellmarker}[1]{%
  \begin{tikzpicture}[remember picture, overlay]
    % based on \spelllevel (argument 1), determine the height the marker should appear at.
    \pgfmathparse{mod(#1-1,10)}% there are 9 levels in core books, but other books may have higher spells. Make sure they do not overshoot the page.
    \pgfmathsetmacro{\markerposition}{\fpeval{2/16 + (1/16 * \pgfmathresult)}}% defines where the first marker shows up (center-point), and the spacing between them
    % print the spelllevel at the edge of the page, \markerposition from the top
    \node [font=\Large, xshift=-0.75cm,yshift=-\markerposition\paperheight] at (current page.north east){\textbf{#1}};
    % draw light-gray lines above and below the spelllevel as guides for the user to apply highlighting (my highlighter's width is 4mm...)
    \draw[lightgray, dotted, line width=1pt] (current page.north east) ++(0cm,-\markerposition\paperheight+8mm) -- ++(-1.25cm,0) -- ++(0,-16mm) -- ++(1.25cm,0);
  \end{tikzpicture}
}
\newenvironment{spellcard}[3]{%
  \begingroup{}   % make macro declarations only affect the content of the "begin" statement
  \def\class{#1}  % character class (wiz, sor, etc.)
  \def\name{#2}   % spell name
  \def\spelllevel{#3}  % spell level for this character class

  % Titlebar for spell (spell name and level)
  \section*{\Huge\name\hfill\MakeUppercase{\class}~\spelllevel}

  \drawspellmarker{\spelllevel}
  \endgroup{}
}{%
  \clearcard% start new spell on a fresh page (ensuring its front face is not printed on the back of the previous card)
  % reset counters after each spell so cards are independent of each other:
  \setcounter{page}{1}
  \setcounter{table}{0}
  \setcounter{figure}{0}
}

% %%%%%%%%%%%%%%%%%%%%%%%%%%%%%%%%%%%%%%%%%%%%%%%%%%%%%%%%%%%%%%%
% Tabular information for spell
% %%%%%%%%%%%%%%%%%%%%%%%%%%%%%%%%%%%%%%%%%%%%%%%%%%%%%%%%%%%%%%%
% command to print one attribute of the spell in the spell card table that can be NULL (shall be omitted if that's the case)
% Note: These cannot be used as the bottom-most line of the table, as they would not print a line after them.
\newcommand{\spellattribute}[2]{%
  \ifthenelse{\equal{#2}{NULL}}{}{\textbf{#1:} & #2 \\ \midrule}
}
% this command is for the last line of each table and cannot be NULL
\newcommand{\spellattributelast}[2]{\textbf{#1:} & #2 \\ \bottomrule}

% Note that TSV headers contain underscores, but LaTeX macros can only contain letters,
% so columns such as "spell_level" should be referred to as "\spelllevel{}"
\newcommand{\spellcardinfo}[1][0.5]{%
  % reduced font size for table so we can fit two next to each other on one card
  \normalsize
  % how much width of the page is offered to the first table is adjustable with optioonal parameter, second table gets the rest:
  \def\firsttablewidth{#1}
  \def\secondtablewidth{\fpeval{1-#1}}
  % longer attribute names should be added to the left table, shorter ones to the right.
  % prevent indenting the tables so we have the full width of the text area available:
  \noindent
  \begin{tabularx}{\firsttablewidth\textwidth}[b]{r>{\raggedright\arraybackslash}X}
    \toprule
    \spellattribute{Casting Time}{\castingtime}
    \spellattribute{Duration}{\duration}
    \spellattribute{Saving Throw}{\savingthrow}
    \spellattributelast{Spell Resist}{\spellresistance}
  \end{tabularx}
  \hfill
  \begin{tabularx}{\secondtablewidth\textwidth-\tabcolsep}[b]{r>{\raggedright\arraybackslash}X}
    \toprule
    \spellattribute{Area}{\area}
    \spellattribute{Range}{\range}
    \spellattribute{Comp.}{\components}
    % evocation school is special as we want to include its descriptor:
    \ifthenelse{\equal{\school}{evocation} \AND\NOT\equal{\descriptor}{NULL}}%
    {\textbf{School:} & \school{} (\descriptor{})} %
    {\textbf{School:} & \school{}                } \\ \bottomrule
  \end{tabularx}
  % A third table spanning the whole width for extra-long attributes:
  \ifthenelse{\equal{\targets}{NULL}\AND\equal{\effect}{NULL}}{}{% omit the table if nothing to display
    \begin{tabularx}{\textwidth}{r>{\raggedright\arraybackslash}X}
      \spellattribute{Target}{\targets}
      \spellattribute{Effect}{\effect}
      % if you add more attributes here, adjust the condition above!
    \end{tabularx}
  }
  % insert some space before the description begins
  \vspace{1ex} \\
  % increased font size for the spell description for easier reading in low light
  \raggedright\Large%
}

% provide a command to print a QR code at the bottom of the current page (front face by default, but user may move it to the back manually)
% drawing multiple QR codes requires a counter:
\newcounter{qrCode}
\AtBeginShipout{\stepcounter{qrCode}\setcounter{qrCode}{0}}
\newcommand{\qrCodeToPrint}{} % define \qrCodeToPrint initially
\newcommand{\spellcardqr}[1]{%
  % we currently only support two qr codes per page, it gets complicated enough as it is.
  % Consider using the back face, even if it's currently empty (use an explicit \clearpage)
  \ifthenelse{\value{qrCode} > 1}{\PackageError{spellcard-templates}%
    {Too many QR codes on one page. Consider moving it to the back of the card with an explicit `clearpage`}%
    {We currently only support two QR codes per page.}%
  }{}%
  % create QR Code itself from the argument to reduce redundancy:
  \def\qrCodeToPrint{\qrcode{#1}}%
  % place the first code of each page at the bottom, opposite the page number.
  % a second code would be placed on the same side as the page number (easier to scan), with enough space to avoid overlap.
  \ifthenelse{\equal{\value{qrCode}}{0}}{%
    \def\qrCodeShift{1cm}% the first qr Code is always placed opposite the page number and needs a smaller offset from the edge of the page
  }{%
    \def\qrCodeShift{4cm}% the second one needs to leave space for the page number
  }%
  \checkoddpage{}\ifoddpage{}%
    % the page number is on the right side
    \ifthenelse{\equal{\value{qrCode}}{0}}{%
      % we want to print the first code on the left side with a smaller offset
      \begin{tikzpicture}[remember picture, overlay]%
        \node [anchor=south west, xshift=\qrCodeShift,yshift=1cm] at (current page.south west) {\qrCodeToPrint};%
      \end{tikzpicture}%
    }{%
      % we want to print the second code on the right side, next to the page number, so need a larger offset
      \begin{tikzpicture}[remember picture, overlay]%
        \node [anchor=south east, xshift=-\qrCodeShift,yshift=1cm] at (current page.south east) {\qrCodeToPrint};%
      \end{tikzpicture}%
    }%
  \else%
    % the page number is on the left side
    \ifthenelse{\equal{\value{qrCode}}{0}}{%
      % we want to print the first code on the right side with a smaller offset
      \begin{tikzpicture}[remember picture, overlay]%
        \node [anchor=south east, xshift=-\qrCodeShift,yshift=1cm] at (current page.south east) {\qrCodeToPrint};%
      \end{tikzpicture}%
    }{%
      % we want to print the second code on the left side, next to the page number, so need a larger offset
      \begin{tikzpicture}[remember picture, overlay]%
        \node [anchor=south west, xshift=\qrCodeShift,yshift=1cm] at (current page.south west) {\qrCodeToPrint};%
      \end{tikzpicture}%
    }%
  \fi%
  \stepcounter{qrCode}%
}

% provide a command for printing a reference chart for spell-level markers
\newcommand{\spellmarkerchart}{%
  \section*{\Huge look-up card for spell-level marker colors}
  \vspace{1ex}
  \raggedright\Large
  To the right you see the markers for all possible spell levels.\\
  These are intended to be colored with a highlighter and serve as a reference
  for which colors you decided to use when printing new cards.
  \drawspellmarker{1}\drawspellmarker{2}\drawspellmarker{3}\drawspellmarker{4}\drawspellmarker{5}
  \drawspellmarker{6}\drawspellmarker{7}\drawspellmarker{8}\drawspellmarker{9}
  \clearcard{}
  % reset page counter the same way the spellcards environment does:
  \setcounter{page}{1}
}

% %%%%%%%%%%%%%%%%%%%%%%%%%%%%%%%%%%%%%%%%%%%%%%%%%%%%%%%%%%%%%%%
% The description of the spell is written in the spells' file within the spellcards environment, no need for a template here.
% %%%%%%%%%%%%%%%%%%%%%%%%%%%%%%%%%%%%%%%%%%%%%%%%%%%%%%%%%%%%%%%


\begin{document}

% include actual spell cards:
% file content generated by convert.sh, meant to be fine-tuned manually (especially the description).

% open a new spellcards environment
\begin{spellcard}{sor}{Fireball}{3}
  % make the data from TSV accessible for to the LaTeX part:
  \newcommand{\name}{Fireball}
  \newcommand{\school}{evocation}
  \newcommand{\subschool}{NULL}
  \newcommand{\descriptor}{fire}
  \newcommand{\spelllevel}{3}
  \newcommand{\castingtime}{1 standard action}
  \newcommand{\components}{V, S, M (a ball of bat guano and sulfur)}
  \newcommand{\costlycomponents}{0}
  \newcommand{\range}{long (400 ft. + 40 ft./level)}
  \newcommand{\area}{20-ft.-radius spread}
  \newcommand{\effect}{NULL}
  \newcommand{\targets}{NULL}
  \newcommand{\duration}{instantaneous}
  \newcommand{\dismissible}{0}
  \newcommand{\shapeable}{0}
  \newcommand{\savingthrow}{Reflex half}
  \newcommand{\spellresistance}{yes}
  \newcommand{\source}{PFRPG Core}
  \newcommand{\verbal}{1}
  \newcommand{\somatic}{1}
  \newcommand{\material}{1}
  \newcommand{\focus}{0}
  \newcommand{\divinefocus}{0}
  \newcommand{\deity}{NULL}
  \newcommand{\SLALevel}{3}
  \newcommand{\domain}{Fire (3)}
  \newcommand{\acid}{0}
  \newcommand{\air}{0}
  \newcommand{\chaotic}{0}
  \newcommand{\cold}{0}
  \newcommand{\curse}{0}
  \newcommand{\darkness}{0}
  \newcommand{\death}{0}
  \newcommand{\disease}{0}
  \newcommand{\earth}{0}
  \newcommand{\electricity}{0}
  \newcommand{\emotion}{0}
  \newcommand{\evil}{0}
  \newcommand{\fear}{0}
  \newcommand{\fire}{1}
  \newcommand{\force}{0}
  \newcommand{\good}{0}
  \newcommand{\languagedependent}{0}
  \newcommand{\lawful}{0}
  \newcommand{\light}{0}
  \newcommand{\mindaffecting}{0}
  \newcommand{\pain}{0}
  \newcommand{\poison}{0}
  \newcommand{\shadow}{0}
  \newcommand{\sonic}{0}
  \newcommand{\water}{0}
  \newcommand{\linktext}{Fireball}
  \newcommand{\id}{207}
  \newcommand{\materialcosts}{NULL}
  \newcommand{\bloodline}{Efreeti (7)}
  \newcommand{\patron}{Elements (6)}
  \newcommand{\mythictext}{The damage dealt increases to 1d10 points of fire damage per caster level (maximum 10d10). Any creature that fails its Reflex saving throw catches on fire (Core Rulebook 444), taking 2d6 points of fire damage each round until the fire is extinguished. Attempts to extinguish this fire use the spell's save DC.}
  \newcommand{\augmented}{Augmented (6\textsuperscript{th}): If you expend two uses of mythic power, the maximum damage increases to 20d10, the area increases to a 40-foot radius spread, and any fire damage dealt by the spell bypasses fire resistance and fire immunity.}
  \newcommand{\hauntstatistics}{NULL}
  \newcommand{\ruse}{0}
  \newcommand{\draconic}{0}
  \newcommand{\meditative}{0}
  \newcommand{\urlenglish}{https://www.d20pfsrd.com/magic/all-spells/f/fireball/}
  \newcommand{\urlsecondary}{http://prd.5footstep.de/Grundregelwerk/Zauber/Feuerball}
  % print the tabular information at the top of the card:
  \spellcardinfo{}
  % draw a QR Code pointing at online resources for this spell on the front face:
  \spellcardqr{\urlenglish}
  \spellcardqr{\urlsecondary}
  %
  % SPELL DESCRIPTION BEGIN
  A \emph{fireball} spell generates a searing explosion of flame that detonates with a low roar
  and deals 1d6 points of fire damage per caster level (maximum 10d6) to every creature within the area.

  Unattended objects also take this damage.
  The explosion creates almost no pressure.

  You point your finger and determine the range (distance and height) at which the \emph{fireball} is to burst.
  A glowing, pea-sized bead streaks from the pointing digit and,
  unless it impacts upon a material body or solid barrier prior to attaining the prescribed range,
  blossoms into the \emph{fireball} at that point.
  An early impact results in an early detonation.
  If you attempt to send the bead through a narrow passage, such as through an arrow slit,
  you must ``hit'' the opening with a ranged touch attack, or else the bead strikes the barrier and detonates prematurely.

  The \emph{fireball} sets fire to combustibles and damages objects in the area.
  It can melt metals with low melting points, such as lead, gold, copper, silver, and bronze.
  If the damage caused to an interposing barrier shatters or breaks through it,
  the \emph{fireball} may continue beyond the barrier if the area permits;
  otherwise it stops at the barrier just as any other spell effect does.
  % SPELL DESCRIPTION END
  %
\end{spellcard}

% this file was modified manually to include the spell description of "Invisibility".

% open a new spellcards environment
\begin{spellcard}{sor}{Invisibility, Greater}{4}
  % make the data from TSV accessible for to the LaTeX part:
  \newcommand{\name}{Invisibility, Greater}
  \newcommand{\school}{illusion}
  \newcommand{\subschool}{glamer}
  \newcommand{\descriptor}{NULL}
  \newcommand{\spelllevel}{4}
  \newcommand{\castingtime}{1 standard action}
  \newcommand{\components}{V, S}
  \newcommand{\costlycomponents}{0}
  \newcommand{\range}{personal or touch}
  \newcommand{\area}{NULL}
  \newcommand{\effect}{NULL}
  \newcommand{\targets}{NULL}% original: {you or creature touched}
  \newcommand{\duration}{1 round/level}
  \newcommand{\dismissible}{1}
  \newcommand{\shapeable}{0}
  \newcommand{\savingthrow}{Will negates (harmless)}
  \newcommand{\spellresistance}{yes (harmless)}
  \newcommand{\source}{PFRPG Core}
  \newcommand{\verbal}{1}
  \newcommand{\somatic}{1}
  \newcommand{\material}{0}
  \newcommand{\focus}{0}
  \newcommand{\divinefocus}{0}
  \newcommand{\deity}{NULL}
  \newcommand{\SLALevel}{4}
  \newcommand{\domain}{NULL}
  \newcommand{\acid}{0}
  \newcommand{\air}{0}
  \newcommand{\chaotic}{0}
  \newcommand{\cold}{0}
  \newcommand{\curse}{0}
  \newcommand{\darkness}{0}
  \newcommand{\death}{0}
  \newcommand{\disease}{0}
  \newcommand{\earth}{0}
  \newcommand{\electricity}{0}
  \newcommand{\emotion}{0}
  \newcommand{\evil}{0}
  \newcommand{\fear}{0}
  \newcommand{\fire}{0}
  \newcommand{\force}{0}
  \newcommand{\good}{0}
  \newcommand{\languagedependent}{0}
  \newcommand{\lawful}{0}
  \newcommand{\light}{0}
  \newcommand{\mindaffecting}{0}
  \newcommand{\pain}{0}
  \newcommand{\poison}{0}
  \newcommand{\shadow}{0}
  \newcommand{\sonic}{0}
  \newcommand{\water}{0}
  \newcommand{\linktext}{Invisibility, Greater}
  \newcommand{\id}{299}
  \newcommand{\materialcosts}{NULL}
  \newcommand{\bloodline}{NULL}
  \newcommand{\patron}{NULL}
  \newcommand{\mythictext}{NULL}
  \newcommand{\augmented}{NULL}
  \newcommand{\hauntstatistics}{NULL}
  \newcommand{\ruse}{0}
  \newcommand{\draconic}{0}
  \newcommand{\meditative}{0}
  \newcommand{\urlenglish}{https://www.d20pfsrd.com/magic/all-spells/i/invisibility/}
  \newcommand{\urlsecondary}{http://prd.5footstep.de/Grundregelwerk/Zauber/Unsichtbarkeit\#h16029-1} % chktex 8
  % print the tabular information at the top of the card:
  \spellcardinfo[0.55]{}
  % draw a QR Code pointing at online resources for this spell on the front face:
  \spellcardqr{\urlenglish}
  \spellcardqr{\urlsecondary}
  %
  % SPELL DESCRIPTION BEGIN
  This spell functions like \emph{invisibility,} except that it doesn't end if the subject attacks:

  The creature or object touched becomes invisible.
  If the recipient is a creature carrying gear, that vanishes, too.
  If you cast the spell on someone else, neither you nor your allies can see the subject,
  unless you can normally see invisible things or you employ magic to do so.

  Items dropped or put down by an invisible creature become visible;
  items picked up disappear if tucked into the clothing or pouches worn by the creature.
  Light, however, never becomes invisible, although a source of light can become so
  (thus, the effect is that of a light with no visible source).
  Any part of an item that the subject carries but that extends more than 10 feet from it becomes visible.

  Of course, the subject is not magically silenced, and certain other conditions
  can render the recipient detectable (such as swimming in water or stepping in a puddle).
  If a check is required, a stationary invisible creature has a +40 bonus on its Stealth checks.
  This bonus is reduced to +20 if the creature is moving.

  \emph{Invisibility} can be made permanent (on objects only) with a \emph{permanency} spell.
  % SPELL DESCRIPTION END
  %
\end{spellcard}

% this file was modified manually to clean up chktex violations in the table generated by pandoc.

% open a new spellcards environment
\begin{spellcard}{sor}{Teleport}{5}
  % make the data from TSV accessible for to the LaTeX part:
  \newcommand{\name}{Teleport}
  \newcommand{\school}{conjuration}
  \newcommand{\subschool}{teleportation}
  \newcommand{\descriptor}{NULL}
  \newcommand{\spelllevel}{5}
  \newcommand{\castingtime}{1 standard action}
  \newcommand{\components}{V}
  \newcommand{\costlycomponents}{0}
  \newcommand{\range}{personal and touch}
  \newcommand{\area}{NULL}
  \newcommand{\effect}{NULL}
  \newcommand{\targets}{you and touched objects or other touched willing creatures}
  \newcommand{\duration}{instantaneous}
  \newcommand{\dismissible}{0}
  \newcommand{\shapeable}{0}
  \newcommand{\savingthrow}{\textbf{none} and Will negates (object)}
  \newcommand{\spellresistance}{\textbf{no} and yes (object)}
  \newcommand{\source}{PFRPG Core}
  \newcommand{\verbal}{1}
  \newcommand{\somatic}{0}
  \newcommand{\material}{0}
  \newcommand{\focus}{0}
  \newcommand{\divinefocus}{0}
  \newcommand{\deity}{NULL}
  \newcommand{\SLALevel}{5}
  \newcommand{\domain}{Travel (5)}
  \newcommand{\acid}{0}
  \newcommand{\air}{0}
  \newcommand{\chaotic}{0}
  \newcommand{\cold}{0}
  \newcommand{\curse}{0}
  \newcommand{\darkness}{0}
  \newcommand{\death}{0}
  \newcommand{\disease}{0}
  \newcommand{\earth}{0}
  \newcommand{\electricity}{0}
  \newcommand{\emotion}{0}
  \newcommand{\evil}{0}
  \newcommand{\fear}{0}
  \newcommand{\fire}{0}
  \newcommand{\force}{0}
  \newcommand{\good}{0}
  \newcommand{\languagedependent}{0}
  \newcommand{\lawful}{0}
  \newcommand{\light}{0}
  \newcommand{\mindaffecting}{0}
  \newcommand{\pain}{0}
  \newcommand{\poison}{0}
  \newcommand{\shadow}{0}
  \newcommand{\sonic}{0}
  \newcommand{\water}{0}
  \newcommand{\linktext}{Teleport}
  \newcommand{\id}{563}
  \newcommand{\materialcosts}{NULL}
  \newcommand{\bloodline}{NULL}
  \newcommand{\patron}{Time (10)}
  \newcommand{\mythictext}{NULL}
  \newcommand{\augmented}{NULL}
  \newcommand{\hauntstatistics}{NULL}
  \newcommand{\ruse}{0}
  \newcommand{\draconic}{0}
  \newcommand{\meditative}{0}
  \newcommand{\urlenglish}{https://www.d20pfsrd.com/magic/all-spells/t/teleport/}
  \newcommand{\urlgerman}{http://prd.5footstep.de/Grundregelwerk/Zauber/Teleportieren}
  % print the tabular information at the top of the card:
  \spellcardinfo[0.575]{}
  % draw a QR Code pointing at online resources for this spell on the front face:
  \spellcardqr{\urlenglish}
  % ATTENTION: URLs for foreign languages cannot be generated and must be provided by you!
  \spellcardqr{\urlgerman}
  % LaTeX-formatted description of the spell, generated from the HTML-formatted description_formatted column:
  Instantly transports to a designated destination no more than 100 miles per caster level distant.
  Interplanar travel is not possible.

  Per three caster levels you can be joined by one willing creature (depending on size).
  All creatures must touch each other and you must touch at least one of them.
  Everyone may bring along as many objects as they can carry.
  Objects held or in use (attended) by another person receive saving throws and spell resistance, creatures do not.

  The accuracy of this spell relies on how well you know the location and layout of the destination.
  Areas of strong physical or magical energy may make teleportation more hazardous or even impossible.

  \begin{longtable}[]{@{}lllll@{}}
    \caption{Roll d\% to determine the outcome}                                                   \\
    \toprule
    Familiarity       & On Target & Off Target & Similar Area & mishap \tabularnewline\midrule
    \endhead{}
    Very familiar     & 01--97    & 98--99     & 100          & --\tabularnewline{}
    Studied carefully & 01--94    & 95--97     & 98--99       & 100\tabularnewline{}
    Seen casually     & 01--88    & 89--94     & 95--98       & 99--100\tabularnewline{}
    Viewed once       & 01--76    & 77--88     & 89--96       & 97--100\tabularnewline{}
    False destination & --        & --         & 81--92       & 93--100\tabularnewline\bottomrule
  \end{longtable}
  \clearpage% explicitly clear page here to prevent underfull vbox warning

  \subsection*{Familiarity}
  \emph{Very familiar} is a place where you have been very often and where you feel at home.

  \emph{Studied carefully} is a place you know well, either because you can currently physically see it or you've been there often.

  \emph{Seen casually} is a place that you have seen more than once but with which you are not very familiar.

  \emph{Viewed once} is a place that you have seen once, possibly using magic such as \emph{scrying.}

  \emph{False destination} does not truly exist or has been so completely altered as to no longer be familiar to you.
  Roll 1d20+80 to obtain results on the table rather than rolling d\%,
  since there is no real destination for you to hope to arrive at or even be off target from.

  \subsection*{Results}
  \emph{On Target}: You appear where you want to be.

  \emph{Off Target}: You appear safely a random distance away from the destination.
  Distance off target is d\% of the distance that was to be traveled.
  The direction off target is determined randomly.

  \emph{Similar Area}: You wind up in an area that's visually or thematically similar to the target area.
  Generally, you appear in the closest similar place within range.
  If no such area exists within the spell's range, the spell simply fails instead.

  \emph{Mishap:} You and anyone else teleporting with you have gotten ``scrambled''.
  You each take 1d10 points of damage, and you reroll on the chart to see where you wind up.
  For these rerolls, roll 1d20+80.

  Each time ``Mishap'' comes up, the characters take more damage and must reroll.

\end{spellcard}

% file content generated by convert.sh, meant to be fine-tuned manually (especially the description).

% open a new spellcards environment
\begin{spellcard}{wiz}{Wall Of Force}{5}
  % make the data from TSV accessible for to the LaTeX part:
  \newcommand{\name}{Wall Of Force}
  \newcommand{\school}{evocation}
  \newcommand{\subschool}{NULL}
  \newcommand{\descriptor}{force}
  \newcommand{\spelllevel}{5}
  \newcommand{\castingtime}{1 standard action}
  \newcommand{\components}{V, S, M (powdered quartz)}
  \newcommand{\costlycomponents}{0}
  \newcommand{\range}{close (25 ft. + 5 ft./2 levels)}
  \newcommand{\area}{NULL}
  \newcommand{\effect}{wall whose area is up to one 10-ft.\ square/level}
  \newcommand{\targets}{NULL}
  \newcommand{\duration}{1 round /level (D)}
  \newcommand{\dismissible}{1}
  \newcommand{\shapeable}{0}
  \newcommand{\savingthrow}{none}
  \newcommand{\spellresistance}{no}
  \newcommand{\source}{PFRPG Core}
  \newcommand{\verbal}{1}
  \newcommand{\somatic}{1}
  \newcommand{\material}{1}
  \newcommand{\focus}{0}
  \newcommand{\divinefocus}{0}
  \newcommand{\deity}{NULL}
  \newcommand{\SLALevel}{5}
  \newcommand{\domain}{Isolation (6)}
  \newcommand{\acid}{0}
  \newcommand{\air}{0}
  \newcommand{\chaotic}{0}
  \newcommand{\cold}{0}
  \newcommand{\curse}{0}
  \newcommand{\darkness}{0}
  \newcommand{\death}{0}
  \newcommand{\disease}{0}
  \newcommand{\earth}{0}
  \newcommand{\electricity}{0}
  \newcommand{\emotion}{0}
  \newcommand{\evil}{0}
  \newcommand{\fear}{0}
  \newcommand{\fire}{0}
  \newcommand{\force}{1}
  \newcommand{\good}{0}
  \newcommand{\languagedependent}{0}
  \newcommand{\lawful}{0}
  \newcommand{\light}{0}
  \newcommand{\mindaffecting}{0}
  \newcommand{\pain}{0}
  \newcommand{\poison}{0}
  \newcommand{\shadow}{0}
  \newcommand{\sonic}{0}
  \newcommand{\water}{0}
  \newcommand{\linktext}{Wall of Force}
  \newcommand{\id}{598}
  \newcommand{\materialcosts}{NULL}
  \newcommand{\bloodline}{NULL}
  \newcommand{\patron}{NULL}
  \newcommand{\mythictext}{The wall's hardness increases to 40, and its hit points increase to 30 per caster level. A non-mythic disintegrate spell or rod of cancellation negates a 10-foot-square section of a mythic wall of force for 1 round, after which the wall reforms at full strength. One side of the wall (chosen by you) repels creatures within 5 feet as a repulsion spell (using the DC wall of force would have if it allowed a saving throw).}
  \newcommand{\augmented}{NULL}
  \newcommand{\hauntstatistics}{NULL}
  \newcommand{\ruse}{0}
  \newcommand{\draconic}{0}
  \newcommand{\meditative}{0}
  % print the tabular information at the top of the card:
  \spellcardinfo{}
  % LaTeX-formatted description of the spell, generated from the HTML-formatted description_formatted column:
  A \emph{wall of force} creates an invisible wall of pure force. The wall
  cannot move and is not easily destroyed. A \emph{wall of force} is
  immune to \emph{dispel magic}, although a mage's disjunction can still
  dispel it.

  A \emph{wall of force} can be damaged by spells as normal, except for
  \emph{disintegrate}, which automatically destroys it. It can be damaged
  by weapons and supernatural abilities, but a \emph{wall of force} has
  hardness 30 and a number of hit points equal to 20 per caster level.
  Contact with a \emph{sphere of annihilation} or \emph{rod of
    cancellation} instantly destroys a \emph{wall of force}.

  Breath weapons and spells cannot pass through a \emph{wall of force} in
  either direction, although \emph{dimension door}, \emph{teleport}, and
  si\-mi\-lar effects can bypass the barrier. It blocks ethereal creatures as
  well as material ones (though ethereal creatures can usually circumvent
  the wall by going around it, through material floors and ceilings).\\
  Gaze attacks can operate through a \emph{wall of force}.

  The caster can form the wall into a flat, vertical plane whose area is
  up to one 10-foot square per level. The wall must be continuous and
  unbroken when formed. If its surface is broken by any object or
  creature, the spell fails.

  \emph{Wall of force} can be made permanent with a \emph{permanency}
  spell.

\end{spellcard}

% file content generated by convert.sh, meant to be fine-tuned manually (especially the description).

% open a new spellcards environment
\begin{spellcard}{wiz}{Reverse Gravity}{7}
  % make the data from TSV accessible for to the LaTeX part:
  \newcommand{\name}{Reverse Gravity}
  \newcommand{\school}{transmutation}
  \newcommand{\subschool}{NULL}
  \newcommand{\descriptor}{NULL}
  \newcommand{\spelllevel}{7}
  \newcommand{\castingtime}{1 standard action}
  \newcommand{\components}{V, S, M/DF (lodestone and iron filings)}
  \newcommand{\costlycomponents}{0}
  \newcommand{\range}{medium (100 ft. + 10 ft./level)}
  \newcommand{\area}{up to one 10-ft.\ cube/level (S)}
  \newcommand{\effect}{NULL}
  \newcommand{\targets}{NULL}
  \newcommand{\duration}{1 round/level (D)}
  \newcommand{\dismissible}{1}
  \newcommand{\shapeable}{1}
  \newcommand{\savingthrow}{none; see text}
  \newcommand{\spellresistance}{no}
  \newcommand{\source}{PFRPG Core}
  \newcommand{\verbal}{1}
  \newcommand{\somatic}{1}
  \newcommand{\material}{1}
  \newcommand{\focus}{0}
  \newcommand{\divinefocus}{1}
  \newcommand{\deity}{NULL}
  \newcommand{\SLALevel}{7}
  \newcommand{\domain}{Void (7)}
  \newcommand{\acid}{0}
  \newcommand{\air}{0}
  \newcommand{\chaotic}{0}
  \newcommand{\cold}{0}
  \newcommand{\curse}{0}
  \newcommand{\darkness}{0}
  \newcommand{\death}{0}
  \newcommand{\disease}{0}
  \newcommand{\earth}{0}
  \newcommand{\electricity}{0}
  \newcommand{\emotion}{0}
  \newcommand{\evil}{0}
  \newcommand{\fear}{0}
  \newcommand{\fire}{0}
  \newcommand{\force}{0}
  \newcommand{\good}{0}
  \newcommand{\languagedependent}{0}
  \newcommand{\lawful}{0}
  \newcommand{\light}{0}
  \newcommand{\mindaffecting}{0}
  \newcommand{\pain}{0}
  \newcommand{\poison}{0}
  \newcommand{\shadow}{0}
  \newcommand{\sonic}{0}
  \newcommand{\water}{0}
  \newcommand{\linktext}{Reverse Gravity}
  \newcommand{\id}{452}
  \newcommand{\materialcosts}{NULL}
  \newcommand{\bloodline}{Starsoul (15)}
  \newcommand{\patron}{Trickery (14)}
  \newcommand{\mythictext}{Creatures in the area or that enter the area must succeed at a Fortitude save or be nauseated.}
  \newcommand{\augmented}{Augmented (8th): If you expend three uses of mythic power, once per round as a move action you may select one secured creature (one that succeeded at its Reflex save) or attached object (such as a tree or cottage) and force it to attempt a Fortitude save against the spell. The selected creature or object can weigh no more than 100 pounds per caster level. If it fails the save, it's pulled free and falls upward.}
  \newcommand{\hauntstatistics}{NULL}
  \newcommand{\ruse}{0}
  \newcommand{\draconic}{0}
  \newcommand{\meditative}{0}
  % print the tabular information at the top of the card:
  \spellcardinfo{}
  % LaTeX-formatted description of the spell, generated from the HTML-formatted description_formatted column:
  This spell reverses gravity in an area, causing unattached objects and
  creatures in the area to fall upward and reach the top of the area in 1
  round. If a solid object (such as a ceiling) is encountered in this
  fall, falling objects and creatures strike it in the same manner as they
  would during a normal downward fall. If an object or creature reaches
  the top of the area without striking anything, it remains there,
  oscillating slightly, until the spell ends. At the end of the spell
  duration, affected objects and creatures fall downward.

  Provided it has something to hold onto, a creature caught in the area
  can attempt a Reflex save to secure itself when the spell strikes.

  Creatures who can fly or levitate can keep themselves from falling.

\end{spellcard}


\end{document}