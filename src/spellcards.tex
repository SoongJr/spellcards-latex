
\documentclass[%
  a4paper,
  twoside,      % one side is the front, the other the back of cards. With \cleardoublepage we can ensure a spell starts on the front of a new card.
  DIV=14,       % reduce margins to account for the reduced space on the smaller cards
  fontsize=12pt,% increase font size to make the text more readable when scaled down to four cards per A4 page
  BCOR=0mm,     % no binding correction necessary
  draft=true,   % helps to point out overfull boxes etc.
]{scrartcl}

\usepackage[utf8]{inputenc}
\usepackage{calc}               % for floating point calculations
\usepackage{ifthen}             % \ifthenelse command
\usepackage{booktabs}           % nicer table formatting options
\usepackage{tabularx}           % for tables that can span the whole page width
\usepackage{amsmath}            % for mathematical symbols (\geq, \leq, ...)
\usepackage{tikz}               % for drawing the spell-level markers
\usepackage{qrcode}             % for generating QR codes
\usepackage{atbegshi}           % for hooking into the shipout process (resetting counters on a new page...)
\usepackage{ifoddpage}          % for checking if a page is odd or even (AKA face is front or back)
% packages used indirectly by pandoc converting from HTML code:
\usepackage{longtable}

% include file to make the cards print four on a page, see README for instructions!
\newcommand{\clearcard}{\clearpage}
% this takes each page of the document, sizes it down to DIN A6 and places it on a DIN A4 page
% in a way that they can be cut and folded over in a way that it all ends up sorted correctly.
% lots of rotation and reordering magic going on.
%\url{https://tex.stackexchange.com/q/638802/86}
%Adapted from \url{http://tex.stackexchange.com/q/279042/86}
\usepackage{pgfmorepages}
\usepackage{atbegshi}           % for using \AtBeginShipout, required to draw lines over all pages

\pgfpagesdeclarelayout{8 on 2, book format}
{%
  \edef\pgfpageoptionheight{\the\paperheight}
  \edef\pgfpageoptionwidth{\the\paperwidth}
  \def\pgfpageoptionborder{0pt}
  \def\pgfpageoptionfirstshipout{1}
}%
{%
  \pgfpagesphysicalpageoptions%
  {%
    logical pages=8,%
    physical pages=2,%
    physical height=\pgfpageoptionheight,%
    physical width=\pgfpageoptionwidth,%
    current logical shipout=\pgfpageoptionfirstshipout%
  }
  \pgfpagesphysicalpage{1}{}
  \pgfpageslogicalpageoptions{1}
  {%
    rotation=0,
    border shrink=\pgfpageoptionborder,%
    resized width=.5\pgfphysicalwidth,%
    resized height=.5\pgfphysicalheight,%
    center=\pgfpoint{.25\pgfphysicalwidth}{.75\pgfphysicalheight}%
  }%
  \pgfpageslogicalpageoptions{3}
  {%
    rotation=180,
    border shrink=\pgfpageoptionborder,%
    resized width=.5\pgfphysicalwidth,%
    resized height=.5\pgfphysicalheight,%
    center=\pgfpoint{.75\pgfphysicalwidth}{.75\pgfphysicalheight}%
  }%
  \pgfpageslogicalpageoptions{5}
  {%
    rotation=0,
    border shrink=\pgfpageoptionborder,%
    resized width=.5\pgfphysicalwidth,%
    resized height=.5\pgfphysicalheight,%
    center=\pgfpoint{.25\pgfphysicalwidth}{.25\pgfphysicalheight},%
  }%
  \pgfpageslogicalpageoptions{7}
  {%
    rotation=180,
    border shrink=\pgfpageoptionborder,%
    resized width=.5\pgfphysicalwidth,%
    resized height=.5\pgfphysicalheight,%
    center=\pgfpoint{.75\pgfphysicalwidth}{.25\pgfphysicalheight},%
  }%
  \pgfpagesphysicalpage{2}{}
  \pgfpageslogicalpageoptions{4}
  {%
    rotation=180,
    border shrink=\pgfpageoptionborder,%
    resized width=.5\pgfphysicalwidth,%
    resized height=.5\pgfphysicalheight,%
    center=\pgfpoint{.25\pgfphysicalwidth}{.75\pgfphysicalheight}%
  }%
  \pgfpageslogicalpageoptions{2}
  {%
    rotation=0,
    border shrink=\pgfpageoptionborder,%
    resized width=.5\pgfphysicalwidth,%
    resized height=.5\pgfphysicalheight,%
    center=\pgfpoint{.75\pgfphysicalwidth}{.75\pgfphysicalheight}%
  }%
  \pgfpageslogicalpageoptions{8}
  {%
    rotation=180,
    border shrink=\pgfpageoptionborder,%
    resized width=.5\pgfphysicalwidth,%
    resized height=.5\pgfphysicalheight,%
    center=\pgfpoint{.25\pgfphysicalwidth}{.25\pgfphysicalheight},%
  }%
  \pgfpageslogicalpageoptions{6}
  {%
    rotation=0,
    border shrink=\pgfpageoptionborder,%
    resized width=.5\pgfphysicalwidth,%
    resized height=.5\pgfphysicalheight,%
    center=\pgfpoint{.75\pgfphysicalwidth}{.25\pgfphysicalheight},%
  }%
}

\AtBeginDocument{
  \pgfpagesuselayout{8 on 2, book format} % use the layout we set up above
}

\AtBeginShipout{
  \AtBeginShipoutUpperLeft{
    % % draw cut lines to show where to cut the paper
    % DISABLED: consumer-level printers are not exact enough to cut on these lines (mine was off by 2mm),
    %           so the front and back sides of cards would not line up!
    % Instead, user should either fold a sacrificial paper in half to use a guide, or use a paper cutter (copyshop? at work?)
    % \begin{tikzpicture}[remember picture, overlay]
    %   \draw[lightgray, dashed] (current page.north east) -- (current page.south east) -- (current page.south west);
    % \end{tikzpicture}
  }
}

% when being cardified, ensure a new card's front face does not start on the backside of the previous card:
\renewcommand{\clearcard}{\cleardoublepage}
% you might want to comment this line while formatting and re-writing spell descriptions so cards will print like normal text, one face per page

% include modern expl3 spell card package
\usepackage{spellcard-expl3}

\begin{document}

% print a look-up card with all spell-level markers user can then apply their chosen color scheme to
% (once you've printed this, you are expected to comment this line)
\spellmarkerchart{}

% include actual spell cards, these class-specific files contain the actual spells to include in the PDF, usually the ones for a character's current level-up.
% It is rare that all spells are needed at once, but if you do wish to do that, here's the command to run:
%   class=sor bash -c 'cd src && find "spells/${class}" -name "*.tex" | sed "s/\.tex$//" | xargs -I{} printf "\\includespell{%s.tex}\n" {} > "spells/${class}.tex"'
%%% ALL SPELLS %%%
% there is always exactly one spell deck without a name (must be first),
% which contains all spells known to this character:
\begin{spelldeck}{}
  \includespell[print]{spells/sor/0/Acid Splash.tex}
  \includespell[print]{spells/sor/0/Dancing Lights.tex}
  \includespell[print]{spells/sor/0/Detect Magic.tex}
  \includespell[print]{spells/sor/0/Ghost Sound.tex}
  \includespell[print]{spells/sor/0/Mage Hand.tex}
  \includespell[print]{spells/sor/0/Prestidigitation.tex}
  \includespell[print]{spells/sor/0/Ray of Frost.tex}
  \includespell[print]{spells/sor/0/Read Magic.tex}
  \includespell[print]{spells/sor/0/Resistance.tex}
  \includespell[print]{spells/sor/1/Burning Hands.tex}
  \includespell[print]{spells/sor/1/Expeditious Retreat.tex}
  \includespell[print]{spells/sor/1/Feather Fall.tex}
  \includespell[print]{spells/sor/1/Grease.tex}
  \includespell[print]{spells/sor/1/Magic Missile.tex}
  \includespell[print]{spells/sor/1/Shocking Grasp.tex}
  \includespell[print]{spells/sor/2/Darkvision.tex}
  \includespell[print]{spells/sor/2/Flaming Sphere.tex}
  \includespell[print]{spells/sor/2/Glitterdust.tex}
  \includespell[print]{spells/sor/2/Knock.tex}
  \includespell[print]{spells/sor/2/Scorching Ray.tex}
  \includespell[print]{spells/sor/2/See Invisibility.tex}
  \includespell[print]{spells/sor/3/Dispel Magic.tex}
  \includespell[print]{spells/sor/3/Fireball.tex}
  \includespell[print]{spells/sor/3/Fly.tex}
  \includespell[print]{spells/sor/3/Protection from Energy.tex}
  \includespell[print]{spells/sor/3/Tongues.tex}
  \includespell[print]{spells/sor/4/Black Tentacles.tex}
  \includespell[print]{spells/sor/4/Elemental Body I.tex}
  \includespell[print]{spells/sor/4/Ice Storm.tex}
  \includespell[print]{spells/sor/4/Invisibility, Greater.tex}
  \includespell[print]{spells/sor/4/Secure Shelter.tex}
  \includespell[print]{spells/sor/5/Elemental Body II.tex}
  \includespell[print]{spells/sor/5/Permanency.tex}
  \includespell[print]{spells/sor/5/Teleport.tex}
  \includespell[print]{spells/sor/5/Transmute Rock to Mud.tex}
  \includespell[print]{spells/sor/5/Wall Of Stone.tex}
  \includespell[print]{spells/sor/6/Chain Lightning.tex}
  \includespell[print]{spells/sor/6/Disintegrate.tex}
  \includespell[print]{spells/sor/6/Elemental Body III.tex}
  \includespell[print]{spells/sor/6/Freezing Sphere.tex}
  \includespell[print]{spells/sor/7/Elemental Body IV.tex}
  \includespell[print]{spells/sor/7/Mage's Magnificent Mansion.tex}
  \includespell[print]{spells/sor/7/Reverse Gravity.tex}
\end{spelldeck}

%%% usage-specific decks %%%
\begin{spelldeck}{damage-dealing}
  \includespell[print]{spells/sor/0/Acid Splash.tex}
  \includespell[print]{spells/sor/0/Ray of Frost.tex}
  \includespell[print]{spells/sor/1/Burning Hands.tex}
  \includespell[print]{spells/sor/1/Magic Missile.tex}
  \includespell[print]{spells/sor/1/Shocking Grasp.tex}
  \includespell[print]{spells/sor/2/Flaming Sphere.tex}
  \includespell[print]{spells/sor/2/Scorching Ray.tex}
  \includespell[print]{spells/sor/3/Fireball.tex}
  \includespell[print]{spells/sor/4/Black Tentacles.tex}
  \includespell[print]{spells/sor/4/Ice Storm.tex}
  \includespell[print]{spells/sor/6/Chain Lightning.tex}
  \includespell[print]{spells/sor/6/Disintegrate.tex}
  \includespell[print]{spells/sor/6/Freezing Sphere.tex}
\end{spelldeck}

\begin{spelldeck}{buffs}
  \includespell[print]{spells/sor/0/Resistance.tex}
  \includespell[print]{spells/sor/1/Expeditious Retreat.tex}
  \includespell[print]{spells/sor/2/Darkvision.tex}
  \includespell[print]{spells/sor/3/Fly.tex}
  \includespell[print]{spells/sor/2/See Invisibility.tex}
  \includespell[print]{spells/sor/3/Protection from Energy.tex}
  \includespell[print]{spells/sor/4/Invisibility, Greater.tex}
\end{spelldeck}

\begin{spelldeck}{utility}
  \includespell[print]{spells/sor/0/Dancing Lights.tex}
  \includespell[print]{spells/sor/0/Detect Magic.tex}
  \includespell[print]{spells/sor/0/Ghost Sound.tex}
  \includespell[print]{spells/sor/0/Mage Hand.tex}
  \includespell[print]{spells/sor/0/Prestidigitation.tex}
  \includespell[print]{spells/sor/0/Read Magic.tex}
  \includespell[print]{spells/sor/1/Feather Fall.tex}
  \includespell[print]{spells/sor/1/Grease.tex}
  \includespell[print]{spells/sor/2/Glitterdust.tex}
  \includespell[print]{spells/sor/2/Knock.tex}
  \includespell[print]{spells/sor/3/Dispel Magic.tex}
  \includespell[print]{spells/sor/3/Tongues.tex}
  \includespell[print]{spells/sor/4/Secure Shelter.tex}
  \includespell[print]{spells/sor/5/Permanency.tex}
  \includespell[print]{spells/sor/5/Teleport.tex}
  \includespell[print]{spells/sor/5/Transmute Rock to Mud.tex}
  \includespell[print]{spells/sor/5/Wall Of Stone.tex}
  \includespell[print]{spells/sor/7/Mage's Magnificent Mansion.tex}
  \includespell[print]{spells/sor/7/Reverse Gravity.tex}
\end{spelldeck}

%% TODO: spell-like abilities

\end{document}